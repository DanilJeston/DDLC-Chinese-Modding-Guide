\documentclass[../Main.tex]{subfiles}
\begin{document}
\chapter*{前言}
Ren'Py是一款开放源代码的自由软件引擎,用来创作透过电脑叙述故事的视觉小说。Ren'Py之名是Ren'ai与Python两词混合而成。Ren'ai为日文,意指“恋爱”,而Python是Ren'Py所使用的语言环境。

{\itshape 《心跳!心跳!文学部!》(Doki Doki Literature Club!},下文简称DDLC)是一款基于Ren'Py编写的游戏。因此,要编写DDLC的 Mod 就必须学习 Ren'Py。

由于Ren'Py与Python密不可分,本书在介绍Ren'Py特性的同时,还讨论了基本Python用法,使二者称为有机的整体。书中介绍了Ren'Py的基本概念,通过简单明了的例子向读者展示Ren'Py的代码与知识。

本书共两部分,分为基础部分与进阶部分,分别介绍了Ren'Py的对话系统、图像显示系统等内容。

本书针对Ren'Py的初学者,从Ren'Py基础知识开始介绍,并在教学过程中会介绍部分Python知识,因此不要求读者有Ren'Py和Python的基础知识,本书中部分内容对于部分读者可能会需要一点的基础的计算机知识,请善用百度。


\subsubsection*{初级教程方法}
在{\itshape C++ Primer Plus}的前言中,有这么一段话:
\begin{quotation}
    大约20年前,{\itshape C Primer Plus}开创了优良的初级教学传统,本书建立在这样的基础之上,吸收了其中很多成功的理念。
    \begin{itemize}
        \item 初级教程应当是友好的、便于使用的指南。
        \item 初级教程不要求您已经熟悉相关的编程概念。
        \item 初级教程强调的是动手学习,通过简短、容易输入的实例阐述一两个概念。
        \item 初级教程用示意图来解释概念。
        \item 初级教程提供问题和联系来检验您对知识的理解,从而适于自学。
    \end{itemize}
    ...

    本书的作者和编辑尽最大的努力使本书简单、明了、生动有趣。我们的目标是,您阅读完本书后,能够编写出可靠、高效的程序,并且觉得这是一种享受。
\end{quotation}
本书秉持着与{\itshape C++ Primer Plus}, {\itshape C Primer Plus}同样的理念,通过简单明了的例子、逻辑连贯的项目式的代码与课后习题来使读者理解、明白Ren'Py的代码与特性。

\subsubsection*{课后练习}
本指南在教授 Ren'Py 知识的同时也会提供课后练习,旨在帮助读者巩固所学知识。课后习题将会单独成册,读者可以在本项目的主页下载课后习题。

\subsubsection*{示例代码}
本书包含大量的示例代码,其中大部分代码是一段完整的、可独立运行的代码。由于 Ren'Py 的特殊性,大部分代码需要放在相应环境中才能运行。本书代码只适用于 Ren'Py 7/8,理论上可以在 Ren'Py 6 上运行,但由于其过时性,本书将不再针对 Ren'Py 6 进行介绍与适配。

\subsubsection*{本书示例代码及注释样式}
为区别普通文本,本书对于实例代码做出以下规定:
\begin{itemize}
    \item 代码英文使用 Hack 字体,中文使用思源等宽字体,字号为 14 点。背景为RBG颜色(235, 235, 235)。如:
    \begin{lstlisting}[numbers=none]
# 这是一行注释
    \end{lstlisting}

    \item 需要您输入的内容将以粗体出现。如:
    \begin{lstlisting}
$ renpy.input()
%\UserInput{22}%
    \end{lstlisting}

    \item 表示代码输出结果的将以斜体出现。如:
    \begin{lstlisting}
>>> 1 + 2
%\Output{3}%
    \end{lstlisting}

    \item 语法中的占位符将用尖括号括起来。您应使用实际的参数、变量等替换占位符。如:
    \begin{lstlisting}[numbers=none]
define <变量名称> = <值:整型、浮点型等>
    \end{lstlisting}
    您应将其替换类似的例子:
    \begin{lstlisting}[numbers=none]
define a = 2
    \end{lstlisting}

    \item 当代码中不含有>>>或...则表示您应该使用文件运行代码,而非Python交互模式。

    \item 本书中只能在Python代码块中运行的语法,将会含有\PyOnly 标签。如下例:


    for循环 \PyOnly
    
    try-except \PyOnly


    同时,本书分为4种注释类型:
    \item 普通注释背景使用25\%色调青色,边框使用75\%青色,如:
    \begin{Comment}
这是一行注释。
    \end{Comment}
    \item 扩展知识背景使用25\%色调绿色,边框使用RGB颜色(105, 190, 78)如:
    \begin{ExtraKnowledge}
    这是一行扩展知识。
    \end{ExtraKnowledge}
    \item 警告背景使用25\%色调黄色,边框使用RGB颜色(150, 150, 0)如:
    \begin{Warning}
    这是一行警告。
    \end{Warning}
    \item 必须注意的内容背景使用25\%色调红色,边框使用75\%红色,如:
    \begin{Attention}
    您必须注意此内容。
    \end{Attention}
\end{itemize}

\subsubsection*{开发本书编程示例所使用的系统}
本书的Ren'Py示例是使用Ren'Py SDK 7.6.1开发的,在Ren'Py SDK 7.8.1、Ren'Py8.0.3、Ren'Py 8.3.7上进行过测试,同时代码在Arch Linux、Windows 11 24H2 64 bit上的进行了测试。

\subsubsection*{获取最新版指南}
\url{https://wwyc.lanzouq.com/b02fb2saj}

密码 :ddlc

\url{https://github.com/DanilJeston/DDLC-Chinese-Modding-Guide}

\subsubsection*{联系方式}
我们的联系邮箱是: \url{team_ninety@outlook.com}。
\newline\newline\par
如果您对本书有任何疑问或建议,请发邮件给我们。若您有兴趣参与本书的编写、完善,可以邮件给我们。同时,若您发现有人未经 CC BY-NC-SA 4.0 方式分发本书,请发邮件给我们。若本书存在部分代码出现错误、无法运行等,请发送邮件给我们。

最后,祝您学习愉快!
\end{document}