\documentclass[../../Main.tex]{subfiles}
\begin{document}
\section{打包模组}
现在,打开Ren'Py Launcher,点击您的项目,在右边您应该可以找到一个选项名为“构建发行版”。点击后,Ren'Py 会自动扫描一遍您的项目。

扫描完成后,您会见到一个有许多选项的界面。首先,确保“选项”中的“向call语句添加from从句,执行一次”勾选了。
\begin{Warning}
    call语句在没有 from 从句的情况下,会有存档损坏的风险。
\end{Warning}
然后,取消“构建分发包”中的“PC:Windows and Linux”与“Macintosh”,勾选“Ren'Py 7 DDLC Compliant Mod”,点击“构建”按钮。稍等片刻,您的模组就会被Ren'Py编译打包成压缩包了。
\begin{Warning}
    请注意,在扫描项目与编译打包过程中,Ren'Py可能会“假死”。假死的时间取决于您电脑的配置与您模组的大小。在假死期间,不要强制关闭Ren'Py,您只需要静待其编译完成后自动恢复。若Ren'Py被强制关闭,其生成的模组很有可能是损坏的,需要重新生成。
\end{Warning}

打包完成后,Ren'Py会自动打开生成的模组的位置。打开压缩包,确保在其game/文件夹下没有audio.rpa、fonts.rpa与images.rpa后,您便可以上传您的模组共其他玩家游玩了。

\begin{Attention}
    根据Team Salvato的IP Guidelines,您应该注意以下几件事:
    \begin{enumerate}
        \item 模组里禁止(MUST NOT)包含DDLC原版内容;
        \item 您不能(NOT ALLOWED)将您的模组上传到任何一个应用商店;
        \item 您不能(MAY NOT)将DDLC、DDLC Plus、您的模组移植到其他平台(事实上,现在有很多的模组都可以在手机上游玩,这原则上违反了IP Guidelines,但是Dan鸽睁一只眼闭一只眼也就过了);
        \item 您的模组不能(NOT ALLOWED)设计在DDLC之前游玩;
        \item 您的模组必须(MUST)是免费的,且在游戏内不能(MAY NOT)包含任何付款或捐赠链接,也不得鼓励玩家在游戏中捐赠、购买商品或捐款。
    \end{enumerate}
    移植模组到手机平台无疑能够扩大DDLC社区。但是若要把模组移植到手机平台,在目前就不得不包含DDLC原版内容。包含DDLC原版内容与移植本身是违反IP Guidelines的(但本指南依然会介绍如何将您的模组移植到手机)。Dan鸽虽然睁一只眼闭一只眼,但这并不意味着我们就能肆意违反IP Guidelines。若您的模组没有移植计划,那么您就不应该制作所谓的“解压即玩”、“傻瓜包”版(即包含原版DDLC内容)。请记住:保护知识产权不仅是保护作者的权利,更是在促进DDLC社区的良好发展。
\end{Attention}
\end{document}