\documentclass[../../Main.tex]{subfiles}
\begin{document}
\section{Ren'Py的界面显示机制}
玩家在Ren'Py游戏中所见到的一切,都可以大致分为两部分:图像(Image)与用户接口(User Interface)。用户接口包括但不限于:对话框、对话角色名、设置菜单、游戏的主界面等。通过基础部分的学习,我们已经学会了使用scene、show和hide语句向玩家展示图像。界面就是Ren'Py为开发者所提供的一种定义用户接口的方式。在本节中,您将会了解Ren'Py的界面显示机制与初步学习界面语言。

界面主要有两个功能:第一,向用户显示信息。信息的可以是文字、图像等。例如,say界面用于向玩家展示对话,包括人物和内容两大部分。第二,允许玩家与游戏交互,例如:按钮。这种界面允许玩家触发某些行为,例如保存游戏与读取存档。

界面主要由组件组成。组件包括:框架(Frame)、窗口(Window)、文本(Text)、按钮(Button)、纵向排列组件(Vbox)等。这些组件可以通过用户接口语句于界面语言中定义。

一般来说,界面将会在以下情况被显式调用或隐式调用:

\begin{itemize}
    \item 通过“show screen”等语句被调用;
    \item Ren'Py底层机制自动调用,如游戏启动时,Ren'Py会自动调用main\_menu界面;
    \item 部分代码隐式调用,如使用menu语句时,会自动调用choice界面;
    \item 将用户的操作与界面绑定,如按下ESC时,调出navigation界面;
\end{itemize}

同时用户在界面上的每次操作(包括移动鼠标),都会使界面刷新。


\end{document}