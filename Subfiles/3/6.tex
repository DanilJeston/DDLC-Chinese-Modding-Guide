\documentclass[../../Main.tex]{subfiles}
\begin{document}
\section{等效语句}
等效语句是在Python中使用Ren'Py语法的一种方式。等效语句只能在Python代码中运行。

\subsection{对话}
Ren'Py的say语句对应两种等效语句。如下例:

\begin{lstlisting}
m "你好!"
\end{lstlisting}

这段代码不仅等效于下列代码:

\begin{lstlisting}
$ m("你好!")
\end{lstlisting}

同时也等效于:

\begin{lstlisting}
$ renpy.say(m, "你好")
\end{lstlisting}

不过,为了保持语义上的通畅,我们常使用后者。

\subsection{图像显示}

\subsubsection{show}

show语句的等效语句的语法如下:

\begin{lstlisting}
renpy.show(<name>, at_list=<position>, zorder=<zorder number>)
\end{lstlisting}

\subsubsection{hide}

hide语句的等效语句的语法如下:

\begin{lstlisting}
renpy.hide(<name>)
\end{lstlisting}

\subsubsection{scene}

scene语句的等效语句的语法如下:

\begin{lstlisting}
renpy.scene()
renpy.show(<name>)
\end{lstlisting}

\subsubsection{with 从句}

with 语句的等效语句的语法如下:

\begin{lstlisting}
renpy.with_statement(<name>)
\end{lstlisting}

\subsection{call 和 jump}

call 和 jump 语句的等效语句的语法如下:

\begin{lstlisting}
renpy.call(script=<script name>)
renpy.jump(screen=<script name>)
\end{lstlisting}

对于更多关于等效语句的介绍,请查阅 \url{https://doc.renpy.cn/zh-CN/statement_equivalents.html}
\end{document}