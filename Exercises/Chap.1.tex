\documentclass[../Exercises.tex]{subfiles}
\begin{document}
    \chapter{第2章 课后练习}
    \section{选择题(每题只有一项符合要求的选项,共6小题,每小题5分,共30分)}
    \begin{question}
        在Ren'Py中,使角色发言的命令是 \paren
        \begin{choices}
            \item say
            \item talk
            \item speak
            \item tell
        \end{choices}
    \end{question}

    \begin{solution}[pre-analysis=【答案】]
        A。
    \end{solution}

    \begin{question}
        若要使莫妮卡发言,下列代码正确的一项是 \paren
        \begin{choices}
            \item m “你好。”
            \item monika \textsf{"}你好。\textsf{"}
            \item m \textsf{"}你好。\textsf{"}
            \item monika “你好。”
        \end{choices}
    \end{question}

    \begin{solution}
        C。在Ren'Py中,文字必须使用英文引号来包住。故A、D选项错误。根据教材表2.1可知,莫妮卡对应的角色编号为m而非monika,故C选项正确,B选项错误。
    \end{solution}

    \begin{question}
        在Ren'Py图像显示指令中,下列指令与其作用对应正确的一项是\paren
        \begin{choices}
            \item scene - 在屏幕上显示任意一个图像
            \item show - 在屏幕上显示任意一个图像
            \item hide - 清空屏幕图像
            \item scene - 清空屏幕图像
        \end{choices}
    \end{question}

    \begin{solution}[pre-analysis=【答案】]
        B。
    \end{solution}

    \begin{question}
        在Ren'Py中,正确表示注释的一项是\paren
        \begin{choices}
            \item /*这是一行注释。*/
            \item \#这是一行注释。
            \item \%这是一行注释。\%
            \item //这是一行注释。
        \end{choices}
    \end{question}

    \begin{solution}[pre-analysis=【答案】]
        B。Ren'Py中使用与Python相同的注释方式,以“\#”开头的代码将不会被执行。
    \end{solution}

    \begin{question}
        在Ren'Py中,如果要在sound通道播放一个名为“oops.ogg”的音频,下列代码正确的一项是\paren
        \begin{choices}
            \item \textsf{play "oops.ogg"}
            \item \textsf{play sound oops.ogg}
            \item \textsf{play sound "oops.ogg"}
            \item \textsf{play oops.ogg}
        \end{choices}
    \end{question}

    \begin{solution}
        C。使用play语句时需要指定通道,A、D选项错误。B选项在Ren'Py中会被视为一个存放在“oops”命名空间的一个名为“ogg”的变量,与题意播放名为“oops.ogg”的音频不符,故错误。
    \end{solution}

    \begin{question}
        下列对于menu语句说法错误的一项是\paren
        \begin{choices}
            \item menu语句用于在屏幕上显示一个选择菜单。
            \item menu语句中的选项必须使用英文引号括起来。
            \item menu语句在显示选项的同时还可以在文本框内显示文字。
            \item menu语句只能有两个选项。
        \end{choices}
    \end{question}

    \begin{solution}[pre-analysis=【答案】]
        D。menu语句可以有无数个选项。
    \end{solution}

\section{综合运用题(根据题目要求,编写出一段符合预期的程序)}
    \begin{problem}
        下列是一段未完成的程序代码,请完成...部分的代码,使得这段代码运行使做到以下要求:
        \begin{enumerate}
            \item 播放名为“club\_day.ogg”的背景音乐;
            \item 显示莫妮卡特殊姿势的立绘;
            \item 莫妮卡说出:“又见到你了,【玩家名字】!”;
            \item 向玩家展示一个选项界面,有两个选项。第一个选项名为“纱世里”,选择后s\_aff加上1;第二个选项名为“夏树”,选择后n\_aff加上1。
        \end{enumerate}
    \end{problem}
\begin{lstlisting}
label ch0_start:
    scene bg club_day
    $ s_aff = 0
    $ n_aff = 0
    ...
    return
\end{lstlisting}

\begin{solution}
    play music "club\_day.ogg"
    show monika 5a
    m "又见到你了,[player]"
    menu:
        "纱世里":
            \$ s_aff += 1
        "夏树":
            \$ n_aff += 1
\end{solution}

\end{document}
