\documentclass[../Exercises.tex]{subfiles}
\begin{document}
\chapter{第3.1章 课后练习}
\section{选择题(每题只有一项符合要求的选项,共6小题,每小题5分,共30分)}
\begin{question}
    在Python中,共有
\end{question}

\begin{solution}[pre-analysis=【答案】]
    A。
\end{solution}

\begin{question}
    若要使莫妮卡发言,下列代码正确的一项是 \paren
    \begin{choices}
        \item m “你好。”
        \item monika \textsf{"}你好。\textsf{"}
        \item m \textsf{"}你好。\textsf{"}
        \item monika “你好。”
    \end{choices}
\end{question}

\begin{solution}
    C。在Ren'Py中,文字必须使用英文引号来包住。故A、D选项错误。根据教材表2.1可知,莫妮卡对应的角色编号为m而非monika,故C选项正确,B选项错误。
\end{solution}

\begin{question}
    在Ren'Py图像显示指令中,下列指令与其作用对应正确的一项是\paren
    \begin{choices}
        \item scene - 在屏幕上显示任意一个图像
        \item show - 在屏幕上显示任意一个图像
        \item hide - 清空屏幕图像
        \item scene - 清空屏幕图像
    \end{choices}
\end{question}

\begin{solution}[pre-analysis=【答案】]
    B。
\end{solution}

\begin{question}
    在Ren'Py中,正确表示注释的一项是\paren
    \begin{choices}
        \item /*这是一行注释。*/
        \item \#这是一行注释。
        \item \%这是一行注释。\%
        \item //这是一行注释。
    \end{choices}
\end{question}

\begin{solution}[pre-analysis=【答案】]
    B。Ren'Py中使用与Python相同的注释方式,以“\#”开头的代码将不会被执行。
\end{solution}

\begin{question}
    在Ren'Py中,如果要在sound通道播放一个名为“oops.ogg”的音频,下列代码正确的一项是\paren
    \begin{choices}
        \item \textsf{play "oops.ogg"}
        \item \textsf{play sound oops.ogg}
        \item \textsf{play sound "oops.ogg"}
        \item \textsf{play oops.ogg}
    \end{choices}
\end{question}

\begin{solution}
    C。使用play语句时需要指定通道,A、D选项错误。B选项在Ren'Py中会被视为一个存放在“oops”命名空间的一个名为“ogg”的变量,与题意播放名为“oops.ogg”的音频不符,故错误。
\end{solution}

\begin{question}
    下列对于menu语句说法错误的一项是\paren
    \begin{choices}
        \item menu语句用于在屏幕上显示一个选择菜单。
        \item menu语句中的选项必须使用英文引号括起来。
        \item menu语句在显示选项的同时还可以在文本框内显示文字。
        \item menu语句只能有两个选项。
    \end{choices}
\end{question}

\begin{solution}[pre-analysis=【答案】]
    D。menu语句可以有无数个选项。
\end{solution}

\end{document}
