\section{特殊脚本}

打开game文件夹,我们能够发现许多rpy文件。在本节中,我们将会了解这些文件内所定义的内容、它们的作用与使用方法。

\subsection{bsod.rpy}

bsod.rpy中定义了DDLC二周目中“电脑蓝屏”的特殊效果。代码会自动检测当前电脑的类型、版本,并生成对应的“死机”效果。

使用方法:
\begin{lstlisting}
call screen bsod()

call screen bsod(bsodCode="<错误代码>", bsodFile="<蓝屏文件,默认libGLESv2.dll>", rsod=False, chinese_screen=True)
\end{lstlisting}

其中,bsodCode是一串伪造的错误代码。默认为DDLC\_ESCAPE\_PLAN\_FAILED。


bsodFile是导致蓝屏的文件名,默认为libGLESv2.dll。


rsod表示是否在将“蓝屏”替换为“红屏”,默认为False。


chinese\_screen表示是否在使用中文版本,默认为True。


\subsection{cgs.rpy}

这个文件中定义了DDLC中的所有CG内容。


\subsection{console.rpy}

这个文件中定义了假结局中莫妮卡删除CG时使用的控制台。其使用方法为:
\begin{lstlisting}
call updateconsole(text="<命令>", history="<提示消息>") # 第一次调用控制台

call updateconsolehistory(text="<提示消息>") # 更新提示消息

call hideconsole  # 隐藏控制台
\end{lstlisting}

例如:
\begin{lstlisting}
call updateconsole(text="print('Hello, World'), history="Hello, World")
\end{lstlisting}

运行代码,我们发现在游戏的左上角出现了控制台,并自动输入了“print('Hello, World')”这串代码,执行结果为“Hello, World”

\subsection{glitchtext.rpy}

glitchtext.rpy中定义了乱码文字。
我们可以通过调用