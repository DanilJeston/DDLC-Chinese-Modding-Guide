\chapter{前言}

Ren'Py是一种视觉小说语言。{\itshape 《心跳!心跳!文学部!》(Doki Doki Literature Club!},下文简称DDLC)正是基于Ren'Py编写的。因此,要编写DDLC的 Mod 就必须学习 Ren'Py。本书旨在通过简单明了的例子教会读者DDLC Mod的开发。另外,由于Ren'Py与Python是密不可分的,本书在介绍 Ren'Py 7的同时也会介绍部分Python知识。
本书中部分内容对于部分读者可能会有一定的编程门槛,请善用百度。

\subsubsection*{实例代码}
本书包含大量的示例代码,由于 Ren'Py 的特殊性,大部分代码需要放在相应环境中才能运行。本书代码只适用于 Ren'Py 7/8,大部分代码理论上可以在 Ren'Py 6 上运行,但由于其过时性,本书将不再针对 Ren'Py 6 进行介绍与适配。

\subsubsection*{本书示例代码及注释样式}
为区别普通文本,本书对于实例代码做出以下规定:
\begin{itemize}
    \item 代码英文使用 Hack 字体,中文使用思源等宽字体,字号为 14 点。背景为 (235, 235, 235)。如:
    \begin{lstlisting}[numbers=none]
# 这是一行注释
    \end{lstlisting}

    \item 需要您输入的内容将以粗体出现。如:
    \begin{lstlisting}
$ renpy.input()
%\UserInput{22}%
    \end{lstlisting}

    \item 表示代码输出结果的将以斜体出现。如:
    \begin{lstlisting}
>>> 1 + 2
%\Output{3}%
    \end{lstlisting}

    \item 语法中的占位符将用尖括号括起来。您应使用实际的参数、变量等替换占位符。如:
    \begin{lstlisting}[numbers=none]
define <变量名称> = <值:整型、浮点型等>
    \end{lstlisting}
    您应将其替换类似的例子:
    \begin{lstlisting}[numbers=none]
define a = 2
    \end{lstlisting}

    \item 当代码中不含有>>>或...则表示您应该使用文件运行代码,而非Python交互模式。

    \item 本书中只能在Python代码块中运行的语法,将会含有\PyOnly 标签。如下例:


    for循环 \PyOnly
    
    try-except \PyOnly


    同时,本书分为4种注释类型:
    \item 普通注释背景使用25\%色调青色,边框使用75\%青色,如:
    \begin{Comment}
这是一行注释。
    \end{Comment}
    \item 扩展知识背景使用25\%色调绿色,边框使用RGB颜色(105, 190, 78)如:
    \begin{ExtraKnowledge}
    这是一行扩展知识。
    \end{ExtraKnowledge}
    \item 警告背景使用25\%色调黄色,边框使用RGB颜色(150, 150, 0)如:
    \begin{Warning}
    这是一行警告。
    \end{Warning}
    \item 必须注意的内容背景使用25\%色调红色,边框使用75\%红色,如:
    \begin{Attention}
    您必须注意此内容。
    \end{Attention}
\end{itemize}

\subsubsection*{获取最新版指南}
\url{https://wwyc.lanzouq.com/b02fb2saj}

密码 :ddlc

\url{https://github.com/DanilJeston/DDLC-Chinese-Modding-Guide}

\subsubsection*{联系方式}
我们的联系邮箱是: \url{team_ninety@outlook.com}。
\newline\newline\par
如果您对本书有任何疑问或建议,请发邮件给我们。若您有兴趣参与本书的编写、完善,可以邮件给我们。同时,若您发现有人未经 CC BY-NC-SA 4.0 方式分发本书,请发邮件给我们。若本书存在部分代码出现错误、无法运行等,请发送邮件给我们。

最后,祝您学习愉快!