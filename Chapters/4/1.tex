\section{增添资源}
一般来说,在Mod工程里的game文件夹下会有一个名叫mod\_assets的文件夹。所有在mod\_assets文件夹里的内容都会在生成发行版时被打包成一个rpa文件。(在后续的章节里,我们将会详细学习如何控制哪些文件打包、哪些不打包,以及如何生成发行版)

\begin{ExtraKnowledge}
    目前,常见的资源获取方法有:从DDLC Community Assets获取(Google Drive)、从一些开发QQ群提供的资源、Reddit等。
\end{ExtraKnowledge}

\begin{Attention}
    在使用资源前,请务必注意版权问题。一般来说,在DDLC Community Assets中的文件只需要在感谢名单中添加作者的名字,但也有部分资源会有更多的要求。请记住:如果没有得到作者的授权就使用资源是侵权违法行为。保护作者的知识产权,不仅是在维护作者的权利、保护作者的心血,更是在保护创造、促进DDLC社区的良好发展。
\end{Attention}


\subsection{定义角色}
在一些模组中,我们常常需要添加一些角色来丰富故事线,或是推动故事情节的发展。要增加角色,我们需要先了解如何定义一个角色。

定义一个角色有两种方式:Character与DynamicCharacter。两种定义方式几乎没有区别,唯一的区别在于Character的名字是固定的,无法更改的;而DynamicCharacter则使用一个变量作为角色名,是动态的,可以更改。由于DDLC中角色都使用DynamicCharacter来定义角色,故本书不会介绍如何使用Character定义,感兴趣者可以前往Ren'Py中文文档了解。

DynamicCharacter的语法如下:

\begin{lstlisting}
define <变量名> = DynamicCharacter('<存储角色名的变量名>', image='<say语句的对话属性图像名>', what_prefix='"', what_suffix='"', ctc="ctc", ctc_position="fixed")
\end{lstlisting}

\begin{Warning}
    对于say语句的对话属性来说,如果image后的图像名错误,那么将会导致其无法使用。所以请务必确保image后的图像名与您定义的角色立绘名一致。对于角色立绘名,请参考\ref{para:3.2.2.1}
\end{Warning}

\begin{ExtraKnowledge}
    在某些情况下,我们可能改变角色说话内容的双引号为其他符号,我们可以通过修改what\_prefix和what\_suffix实现效果。

    正常情况下,除旁白外所有的角色说的话都会被双引号包住。如:
    \begin{lstlisting}
m "你好"
    \end{lstlisting}

    那么在游戏中的效果为:“你好”。如果我们想要修改为:「你好」,那么只需要找到莫妮卡角色的定义,并将what\_prefix改为'「', what\_suffix改为'」'即可实现。实际代码如下:
    \begin{lstlisting}
define m = DynamicCharacter('m_name', image='monika', what_prefix='「', what_suffix='」', ctc='ctc', ctc_position='fixed')
    \end{lstlisting}
\end{ExtraKnowledge}

如我现在需要定义一个名为Charlie的角色,其使用的图像名为charlie,存储角色名的变量名叫c\_name:
\begin{lstlisting}
define c_name = "Charlie"
define c = DynamicCharacter('c_name', image='charlie', what_prefix='"', what_suffix='"', ctc="ctc", ctc_position="fixed")
\end{lstlisting}

现在,我们就成功定义了Charlie这个角色。那么这段代码应该放在哪里呢?答案是game目录下的definitions.rpy文件内。这个文件里储存着所有在游戏中需要用到的资源的定义。

打开编辑器的查找功能,搜索:“define s = DynamicCharacter”,随后在这一行上面或下面增添例如上述的代码。

现在,definitions.rpy文件内的代码应该长这样:
\begin{lstlisting}
    # 角色变量

    define narrator = Character(ctc="ctc", ctc_position="fixed")
    define mc = DynamicCharacter('player', what_prefix='"', what_suffix='"', ctc="ctc", ctc_position="fixed")
    define s = DynamicCharacter('s_name', image='sayori', what_prefix='"', what_suffix='"', ctc="ctc", ctc_position="fixed")
    define m = DynamicCharacter('m_name', image='monika', what_prefix='"', what_suffix='"', ctc="ctc", ctc_position="fixed")
    define n = DynamicCharacter('n_name', image='natsuki', what_prefix='"', what_suffix='"', ctc="ctc", ctc_position="fixed")
    define y = DynamicCharacter('y_name', image='yuri', what_prefix='"', what_suffix='"', ctc="ctc", ctc_position="fixed")
    define ny = Character('夏树 & 优里', what_prefix='"', what_suffix='"', ctc="ctc", ctc_position="fixed")
    define c = DynamicCharacter('c_name', image='charlie', what_prefix='"', what_suffix='"', ctc="ctc", ctc_position="fixed")
    
    # ...
    
    # Default Name Variables
    default s_name = "纱世里"
    default m_name = "莫妮卡"
    default n_name = "夏树"
    default y_name = "优里"
    default c_name = "Charlie"
    
    
\end{lstlisting}

\subsection{增加、定义图片}

现在,在mod\_assets文件夹下创造一个名为images的文件夹。在后续的教程中,我们将会把所有的图片资源都存储在这个images文件夹内。

\begin{Warning}
请注意,对于所有图片,其格式都应为PNG,且背景应该是透明的,图像编号不能和已有的重复。
角色立绘资源的分辨率应为960x960以保证兼容性,且对于一个完整的立绘应遵循原版DDLC的原则分成三个图像:头部、左半身、右半身。
背景图像的尺寸应为16:9,分辨率应该为1280x720。如果图像资源分辨率过大,您可以使用软件降低图像分辨率或使用size属性解决:
\begin{lstlisting}
size (1280,720) # 添加 size 属性
\end{lstlisting}

为了兼容性,我们建议您新建的所有文件(夹)名称全部为英文小写字母,且不使用中文。
\end{Warning}

\subsubsection{角色立绘}

在获取角色立绘后,在images文件夹下创建一个文件夹,名为您想要添加立绘的角色名,如sayori。将图像放进您刚才创建的文件夹内,同时我们建议您将图像名改为26个小写字母或0-9数字中的任意一个(建议从a开始到z依次排列并遵循\ref{para:3.2.2.1}中的定义)。

增添图像后,接下来我们就该定义角色立绘了。定义图像的语法如下:

\begin{lstlisting}
image <图像名> = "<资源地址>"
\end{lstlisting}

使用image定义后的图像就可以被show语句使用了。


由于DDLC将角色立绘拆分成为了三部分(请参考\ref{para:3.2.2.1}),故对于角色立绘的定义会非常的复杂。

您可以按照以下步骤判断选取相应的代码进行修改:
\begin{enumerate}
    \item 您添加的立绘是DDLC原有角色:
    \begin{enumerate}
        \item 您只添加了表情:
        \begin{lstlisting}[numbers=none]
image <角色名> <立绘编号> = im.Composite((960, 960), (0, 0), "<角色名>/<数字>l.png", (0, 0), "<角色名>/<数字>r.png", (0, 0), "mod_assets/images/<角色名>/<图像资源>.png") 
        \end{lstlisting}

        \item 您只添加了身体姿势:
        \begin{lstlisting}[numbers=none]
image <角色名> <立绘编号> = im.Composite((960, 960), (0, 0), "mod_assets/images/<角色名>/<数字>l.png", (0, 0), "mod_assets/images/<角色名>/<数字>l.png", (0, 0), "<角色名>/<字母>l.png") 
        \end{lstlisting}

        \item 您同时添加了表情与身体姿势:
        \begin{lstlisting}[numbers=none]
image <角色名> <立绘编号> = im.Composite((960, 960), (0, 0), "mod_assets/images/<角色名>/<数字>l.png", (0, 0), "mod_assets/images/<角色名>/<数字>l.png", (0, 0), "mod_assets/images/<角色名>/<图像资源>.png") 
        \end{lstlisting}
    \end{enumerate}

    \item 您添加的立绘是新定义的的角色:
    \begin{lstlisting}[numbers=none]
image <角色名> <立绘编号> = im.Composite((960, 960), (0, 0), "mod_assets/images/<角色名>/<数字>l.png", (0, 0), "mod_assets/images/<角色名>/<数字>l.png", (0, 0), "mod_assets/images/<角色名>/<图像资源>.png") 
    \end{lstlisting}
\end{enumerate}

例如现在为优里增添了新的服装与表情,且图像资源名分别为 1l, 1r, happy.png
那么您使用的代码应该长这个样子:
\begin{lstlisting}
# m means mod
images yuri m1a = im.Composite((960, 960), (0, 0), "mod_assets/images/yuri/1l.png", (0, 0), "mod_assets/images/yuri/1r.png", (0,0), "mod_assets/images/yuri/happy.png")
\end{lstlisting}


对于角色立绘的列举,您可以前往 \url{https://docs.dokimod.top/pages/0a59cf/} 查看。

\subsubsection{背景图像}
在获取背景图像后,在images文件夹下创建一个名为bg文件夹。将背景放进您刚才创建的文件夹内,同时我们建议您将图像名改为小写的单词。

增添背景后,接下来我们就该定义背景了。定义背景的语法如下:

\begin{lstlisting}
image bg <背景名> = "mod_assets/images/bg/<背景图像资源名>.png"
\end{lstlisting}

\begin{ExtraKnowledge}
    这里的bg指background,它实际上是图像名的一部分,其目的是为了区分角色立绘与背景立绘。
\end{ExtraKnowledge}

\subsection{音频资源}
现在,在mod\_assets文件夹下创造一个名为audio的文件夹。将音频放进您刚才创建的文件夹内,同时我们建议您将音频名改为小写的单词。

\begin{Warning}
    Ren'Py 只支持以下几种音频格式:
    \begin{itemize}
        \item Opus
        \item Ogg
        \item MP3
        \item WAV
    \end{itemize}
\end{Warning}

然后,打开definitions.rpy文件,新增以下代码:
\begin{lstlisting}
define audio.<音频名> = "mod_assets/audio/<音频文件名>.<后缀名>"
\end{lstlisting}

请注意,音频名不能和已有的定义重复,否则会覆盖定义。

\subsubsection{节选播放}
Ren'Py支持将音频切割成一段播放。其中,共有三种行为:
\begin{itemize}
    \item from
    \item to
    \item loop
\end{itemize}

\paragraph{from to} from与to标签用于指定播放文件的起始位置和终止位置。其基本用法为:
\begin{lstlisting}
<from [开始] to [结束]>
\end{lstlisting}
例如,现在有一个音频名为“festival.ogg”存放在mod\_assets文件夹中。现在我们想要从这个音频的第5秒开始播放,到第15秒停止,我们可以修改音频的定义为:

\begin{lstlisting}
define festival = "<from 5 to 15>mod_assets/festival.ogg"
\end{lstlisting}

这样,在播放festival音频时就会从第5秒开始,到第15秒结束。

\paragraph{loop} loop标签用于指定音频循环播放。其基本用法为:

\begin{lstlisting}
<loop [开始,默认为0]>
\end{lstlisting}

例如,现在有一个音频名为“music.ogg”,我们想要从第10秒开始播放,并循环播放这个音频,可以修改音频的定义为:

\begin{lstlisting}
define bg_music = "<loop 10>mod_assets/music.ogg"
\end{lstlisting}

