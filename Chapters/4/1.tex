\section{增添资源}
一般来说,在Mod工程里的game文件夹下会有一个名叫mod\_assets的文件夹。所有在mod\_assets文件夹里的内容都会在生成发行版时被打包成一个rpa文件。(在后续的章节里,我们将会详细学习如何控制哪些文件打包、哪些不打包,以及如何生成发行版)

\begin{ExtraKnowledge}
    目前,常见的资源获取方法有:从DDLC Community Assets获取(Google Drive)、从一些开发QQ群内获取打包好的DDLC Communi Assets、Reddit等。
\end{ExtraKnowledge}

\begin{Attention}
    在使用资源前,请务必注意版权问题。一般来说,在DDLC Community Assets中的文件只需要在感谢名单中添加作者的名字,但也有部分资源会有更多的要求。请记住:在一个资源必须得到作者授权的情况下,没有得到授权就使用该资源的行为是侵权行为。
\end{Attention}


\subsection{定义角色}
在一些模组中,我们常常需要添加一些角色来丰富故事线,或是推动故事情节的发展。要增加角色,我们需要先了解如何定义一个角色。

定义一个角色有两种方式:Character与DynamicCharacter。两种定义方式几乎没有区别,唯一的区别在于Character的名字是固定的,无法更改的;而DynamicCharacter则使用一个变量作为角色名,是动态的,可以更改。由于DDLC中角色都使用DynamicCharacter来定义角色,故本书不会介绍如何使用Character定义,感兴趣者可以前往Ren'Py中文文档了解。

DynamicCharacter的语法如下:

\begin{lstlisting}
define <变量名> = DynamicCharacter('<存储角色名的变量名>', image='<say语句的对话属性图像名>', what_prefix='"', what_suffix='"', ctc="ctc", ctc_position="fixed")
\end{lstlisting}

\begin{Warning}
    对于say语句的对话属性来说,如果image后的图像名错误,那么将会导致其无法使用。所以请务必确保image后的图像名与您定义的角色立绘名一致。对于角色立绘名,请参考\ref{par:3.2.1}
\end{Warning}

\begin{ExtraKnowledge}
    在某些情况下,我们可能改变角色说话内容的双引号为其他符号,我们可以通过修改what\_prefix和what\_suffix实现效果。

    正常情况下,除旁白外所有的角色说的话都会被双引号包住。如:
    \begin{lstlisting}
m "你好"
    \end{lstlisting}

    那么在游戏中的效果为:"你好"。如果我们想要修改为:【你好】,那么只需要找到莫妮卡角色的定义,并将what\_prefix改为'【', what\_suffix改为'】'即可实现。实际代码如下:
    \begin{lstlisting}
define m = DynamicCharacter('m_name', image='monika', what_prefix='【', what_suffix='】', ctc='ctc', ctc_position='fixed')
    \end{lstlisting}
\end{ExtraKnowledge}

如我现在需要定义一个名为Charlie的角色,其使用的图像名为charlie,存储角色名的变量名叫c\_name:
\begin{lstlisting}
define c_name = "Charlie"
define c = DynamicCharacter('c_name', image='charlie', what_prefix='"', what_suffix='"', ctc="ctc", ctc_position="fixed")
\end{lstlisting}

现在,我们就成功定义了Charlie这个角色。那么这段代码应该放在哪里呢?答案是game目录下的definitions.rpy文件内。这个文件里储存着所有在游戏中需要用到的资源的定义。

打开编辑器的查找功能,搜索:“define s = DynamicCharacter”,随后在这一行上面或下面增添例如上述的代码。

现在,definitions.rpy文件内的代码应该长这样:
\begin{lstlisting}
    # 角色变量

    define narrator = Character(ctc="ctc", ctc_position="fixed")
    define mc = DynamicCharacter('player', what_prefix='"', what_suffix='"', ctc="ctc", ctc_position="fixed")
    define s = DynamicCharacter('s_name', image='sayori', what_prefix='"', what_suffix='"', ctc="ctc", ctc_position="fixed")
    define m = DynamicCharacter('m_name', image='monika', what_prefix='"', what_suffix='"', ctc="ctc", ctc_position="fixed")
    define n = DynamicCharacter('n_name', image='natsuki', what_prefix='"', what_suffix='"', ctc="ctc", ctc_position="fixed")
    define y = DynamicCharacter('y_name', image='yuri', what_prefix='"', what_suffix='"', ctc="ctc", ctc_position="fixed")
    define ny = Character('夏树 & 优里', what_prefix='"', what_suffix='"', ctc="ctc", ctc_position="fixed")
    define c = DynamicCharacter('c_name', image='charlie', what_prefix='"', what_suffix='"', ctc="ctc", ctc_position="fixed")
    
    # ...
    
    # Default Name Variables
    default s_name = "纱世里"
    default m_name = "莫妮卡"
    default n_name = "夏树"
    default y_name = "优里"
    default a_name = "Charlie"
    
    
\end{lstlisting}

\subsection{增加、定义图片}

现在,在mod\_assets文件夹下创造一个名为images的文件夹。在后续的教程中,我们将会把所有的图片资源都存储在这个images文件夹内。

\begin{Warning}
图像资源的格式应该为PNG,且背景应该是透明的。
角色图像资源的分辨率应为 960x960 以保证兼容性。
背景图像的分辨率应该为 1280x720,否则图片在显示时会出现意料之外的状况。如果背景图片的分辨率超过上述分辨率,至少您的图片也应该为 16:9 的尺寸,这种情况下可以使用增添size属性解决问题:

\begin{lstlisting}
size (1280,720) # 添加 size 属性
\end{lstlisting}

为了兼容性,我们建议您新建的所有文件(夹)名称全部为英文小写字母,且不使用中文。
\end{Warning}

\subsubsection{角色图片}

在获取角色资源图片后,
