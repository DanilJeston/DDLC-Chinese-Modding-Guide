\section{界面语言}
界面语言是定义一个用户接口最正式的办法。它包含定义新界面的语句(screen语句)、添加可视组件至界面的语句和控制型语句。

\subsection{初步了解界面语言的结构}

以模板内定义的say界面为例:
\begin{lstlisting}
screen say(who, what):
    style_prefix "say"

    window:
        id "window"

        text what id "what"

        if who is not None:

            window:
                style "namebox"
                text who id "who"

    # 如果有对话框头像,会将其显示在文本之上。请不要在手机界面下显示这个,因为
    # 没有空间。
    if not renpy.variant("small"):
        add SideImage() xalign 0.0 yalign 1.0

    use quick_menu
\end{lstlisting}

首先,第1行的“screen say(who, what):”定义了一个名为say的界面,并且该界面在调用时接受两个参数:who与what。screen是最基础也是最重要的一个界面语言语句,用于定义新界面。screen语句接受一个参数,即界面名,同时可以在界面名后用英文括号定义这个界面所接受的参数。转换为代码如下形式:
\begin{lstlisting}
# 接受参数
screen <界面名>(<参数1>, <参数2>, ...):
    ...
# 不接受参数
screen <界面名>:
    ...
\end{lstlisting}

第2行的“style\_prefix "say"”,是screen语句的一个特性,包括特性名称和特性值,特性值是简单表达式。“style\_prefix”是特性名称,“"say"”则是特性值。该语句表示在该界面内的所有组件的样式(style),都以“say”为前缀。例如,第七行的“text”组件,将会默认使用“say\_text”这个样式,除非通过style参数重新指定一个样式。

\begin{ExtraKnowledge}
    事实上,该组件所用样式应为say\_dialogue,但style\_prefix效果确实如上述所说。我们推测是由于修改组件id特性为“what”,Ren'Py内部使用say\_dialogue替代say\_text。
\end{ExtraKnowledge}

第4、7、11、13、18行,分别向这个界面中添加了不同的组件。由此可见,大多数界面语言语句使用通用的语法。语句的开头以某个关键词进入。如果语句使用参数,参数就跟着开头的关键词后面,参数列表是使用空格分隔的简单表达式。同时,后面一般会跟一个特性(property)列表,多个同级特性组成一个特性列表,特性列表中各个特性使用空格分隔。例如,第7行的text组件,将参数what作为text组件的参数,id则是text组件的一个特性,"what"代表着这是这个特性的值。


第1、4、11行,screen语句和window组件后面跟着一个英文冒号,代表该组件是一个语句块(Block)。多个同一等级的语句组成一个语句块。语句块的内容可以是以下内容:

\begin{itemize}
    \item 特性列表;
    \item 界面语言。
\end{itemize}

例如,第1行后的所有组件,都是screen语句下的代码块;第7行的text组件,是在window组件下的语句块,可以理解为text组件被包含在window组件内。
