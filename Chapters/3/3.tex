\section{在Ren'Py中使用Python语句}
\label{sec:4.3}
在Ren'Py中有多种方式可以使用Python语句:Python语句块(block)、单行Python语句、init python语句。

\subsection{Python语句块}
Python语句块是使用Python的最方便的一种形式。一个Python语句块包含两个部分:
\begin{itemize}
    \item 开头的声明
    \item Python代码
\end{itemize}

如下例:

\begin{lstlisting}
# Chapter 0
label ch0_start:
    python:
        # affection: 好感度
        n_aff = 0
        m_aff = 0
    scene bg club_day
    "{cps=20}快到{b}学园祭{/b}了。{/cps}{w=.5}{nw}"
    show monika 1a at l41 zorder 1
    m "各位!我们得开始准备了!"
    show sayori 1a at h42 zorder 1
    s "好耶!!!!"
    show natsuki 1a at t43 zorder 1
    n 2d "啊,我都等不及学园祭了。"
    n "肯定会很棒的!"
    show yuri 1a at s44 zorder 1
    y "..."
    scene bg club_day
    show monika 2a at t21 zorder 1
    show sayori 2a at t22 zorder 1
    m "那么,是时候来进行分工了。"
    m 4k "[player],你想要做什么? "
    menu:
        "做小蛋糕":
            s "夏树的小蛋糕最好吃了!"
            python:
                $ n_aff += 1
                $ m_aff -= 1
        "布置教室":
            m "那我们可得抓紧时间了!"
            python:
                $ m_aff += 1
                $ n_aff -= 1
    return

\end{lstlisting}

当出现多个Python语句块时,Ren'Py会根据先后顺序依次执行。同时,我们可以使用hide与in关键词来改变Python语句块的行为。

hide关键词会使Python语句块在一个独立的环境(命名区)下运行,即不与其他Python语句块共用一个环境。在具有hide关键词的Python语句块中的所有变量将不会被保存。

in关键词则可以让Python语句块在一个独立的环境下运行。其他环境可以通过“储存区.变量”来使用其他环境的内容。

如下例:
\begin{lstlisting}
python:
    a = 2

python hide:
    a = 1

python in c:
    a = 4

python:
    print(c.a)
    print(a)
\end{lstlisting}

最后的运行结果为:4、2。

\subsection{单行Python语句}
大多数情况下,我们只有一行Python语句需要执行。此时使用Python语句块无疑会对开发者造成多余的输入。为了让编写只有一行的Python更方便快捷,Ren'Py提供了单行Python语句。

单行Python语句以美元符号(\$)开头。如:
\begin{lstlisting}
$ s_aff = 0

$ winner = 'monika'
\end{lstlisting}

\subsection{init python语句}
init python语句在Ren'Py初始化阶段运行,会早于其他代码。这种功能可以用于定义类和函数或者配置变量。
\begin{ExtraKnowledge}
    在init与python之间还可以放一个运行优先级,默认为0。init语句将会按照从低到高的顺序执行。即优先级为-1的代码会优先于优先级为0的代码执行。在优先级相同的情况下,Ren'Py会根据本代码所在文件的Unicode码顺序执行。即在代码优先级都在0的情况下,a.rpy内的代码总会优先于b.rpy内的代码执行。
\end{ExtraKnowledge}

\begin{Attention}
    为了避免与Ren'Py自身代码冲突,您只应使用-999到999范围内作为优先级。同时,原则上除特殊代码外,init优先级应全部大于等于0。
\end{Attention}

init python语句也可以使用hide或in分句,与普通python语句用法相同。

在init python语句中的变量不会被存档。因此,在init语句中定义的应为常量。

\begin{lstlisting}
# Chapter 0
init -1 python:
    demo = True

label ch0_start:
    scene bg club_day
    "{cps=20}快到{b}学园祭{/b}了。{/cps}{w=.5}{nw}"
    show monika 1a at l41 zorder 1
    m "各位!我们得开始准备了!"
    show sayori 1a at h42 zorder 1
    s "好耶!!!!"
    show natsuki 1a at t43 zorder 1
    n 2d "啊,我都等不及学园祭了。"
    n "肯定会很棒的!"
    show yuri 1a at s44 zorder 1
    y "..."
    scene bg club_day
    show monika 2a at t21 zorder 1
    show sayori 2a at t22 zorder 1
    m "那么,是时候来进行分工了。"
    m 4k "[player],你想要做什么? "

    if demo:
        "Demo 版剧情到此结束。"
        return

    menu:
        "做小蛋糕":
            s "夏树的小蛋糕最好吃了!"
            python:
                $ n_aff += 1
                $ m_aff -= 1
        "布置教室":
            m "那我们可得抓紧时间了!"
            python:
                $ m_aff += 1
                $ n_aff -= 1
    return

\end{lstlisting}

\begin{ExtraKnowledge}
    Python语句块还可以使用early分句。在Ren'Py的运行生命周期中,python early是被最早运行的。python early中的代码将会早于所有代码运行,因此,python early语句适合用来修改Ren'Py底层处理机制。但由于修改底层处理代码可能会导致一系列问题,我们不会在这里展开讲解。有兴趣者可以前往 \url{https://doc.renpy.cn/zh-CN/lifecycle.html} 了解python early语句的使用。
\end{ExtraKnowledge}

