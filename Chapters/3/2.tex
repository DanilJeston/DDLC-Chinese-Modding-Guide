\section{函数与类\PyOnly }

\subsection{函数}
在编程中,我们往往会重复执行一段代码或进行类似的操作。为了减少代码的重复,我们可以使用函数。函数的作用就是把相对独立的某个功能抽象出来,成为一个独立的个体。

\subsubsection{函数的定义}
定义一个函数,只需要开头为def即可。如下例:
\begin{lstlisting}
def test(arg1, arg2):
    print("Arg1 is: " + arg1)
    print("Arg2 is: " + arg2)
    return
\end{lstlisting}

其中,test为这个函数的名字,arg1、arg2则为这个函数接受的参数。若留空,在代表该函数不接受参数。引号后的部分被称为函数主体,是调用该函数后具体的一些代码。return语句则是函数运行成功后返回的值,可以留空。

调用函数也非常简单,如下例:
\begin{lstlisting}
test(1, 2)
%\Output{Arg1 is: 1}%
%\Output{Arg2 is: 2}%

test(5, arg2=1)
%\Output{Arg1 is: 5}%
%\Output{Arg2 is: 1}%

test(arg2=4, arg1=2)
%\Output{Arg1 is: 2}%
%\Output{Arg2 is: 4}%

test(arg1=7, arg2=-7)
%\Output{Arg1 is: 7}%
%\Output{Arg2 is: -7}%
\end{lstlisting}

由此可见,在给函数传递参数时,可以直接传递,也可以使用“参数名=参数”的方式传递。

\subsubsection{函数命名空间}
所有在函数中的变量,都位于一个独立的命名空间内。该命名空间只能在该函数内使用,函数外的代码都无法读取或修改函数命名空间内的变量。同时,函数内也无法直接修改全局命名空间的变量。

如下例:
\begin{lstlisting}
>>> a = 1
>>> def test():
...     a = 2
...     print(a)
>>> print(a)
%\Output{1}%
>>> test()
%\Output{2}%
>>> print(a)
%\Output{1}%
\end{lstlisting}

如果要在函数内修改全局变量,则应使用global语句声明要使用的变量,如下例:
\begin{lstlisting}
>>> a = 1
>>> def test():
...     global a
...     a = 2
...     print(a)


>>> print(a)
%\Output{1}%
>>> test()
%\Output{2}%
>>> print(a)
%\Output{2}%
\end{lstlisting}

\subsection{类}
Python是一门面向对象的语言,而类是一种用来描述具有相同属性和方法(函数)的对象的集合。它定义了该集合中每个对象共同具有的方法。对象是类的实例化。

\subsubsection{类的定义}
定义一个类,只需要开头为class即可。如下例:

\begin{lstlisting}
class Test:
    def __init__(self):
        self.a = 1

    def counter(self):
        self.a += 1
        print(self.a)
\end{lstlisting}

上述例子中,定义了一个名为Test的类,这个类中有一个变量为a,且有一个counter方法用于打印a的值。
\begin{ExtraKnowledge}
    在类中,以双下划线开头的、具有特殊的方法名的方法叫魔法方法(Magic Methods)。上述例子中的\_\_init\_\_特殊方法用于在实例话一个类时会运行的初始化函数。类似的魔法方法还有\_\_eq\_\_,\_\_ne\_\_等。

    self是一个特殊的参数。当类被实例化后,无论调用其中的哪一个方法,Python都会给第一个参数传递这个对象自己,且第一个参数一定指向这个对象。
\end{ExtraKnowledge}

\subsubsection{类的使用}
在使用类前,需要对类进行实例化。实例化后,类就会变成对象。创造对象和创造变量类似。如下例:

\begin{lstlisting}
>>> test = Test()
>>> test.counter()
%\Output{2}%
>>> test.counter()
%\Output{3}%
\end{lstlisting}

第一行的“test = Test()”创造了一个Test对象。剩下两行代码则是在调用这个对象的counter方法。

如果要在类中如果要定义或使用一个属性,必须使用“self.”的方式进行赋值,否则这个属性就只会存在于方法的命名空间而不是对象的命名空间。

\begin{ExtraKnowledge}
    Python中类与函数的使用远远不止这些,您可以前往\url{https://www.runoob.com/python3/python3-function.html}与\url{https://www.runoob.com/python3/python3-class.html}了解更多。
\end{ExtraKnowledge}
