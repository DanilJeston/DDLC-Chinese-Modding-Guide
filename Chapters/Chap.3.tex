\chapter{Python与Ren'Py}
\begin{ChapterGoals}
    \begin{itemize}
        \item 了解Python的基本数据类型;
        \item 学会使用函数与类;
        \item 学会在Ren'Py代码中嵌入Python代码;
        \item 学会使用变量;
        \item 学会进行判断控制;
        \item 学会使用等效语句;
    \end{itemize}
\end{ChapterGoals}

Ren'Py与Python是一个不可分割的整体,在Ren'Py中,许多复杂的操作都需要依靠Python来完成。本章我们将会初步学习如何在Ren'Py中使用Python语句。

\begin{Attention}
    本章只会浅显的介绍Python的用法,若要真正的学习Python,您可以前往 \url{https://www.runoob.com/python3/python3-tutorial.html} 进行更深入的学习。切记,本书不以Python为主。
\end{Attention}

\begin{Warning}
    如果您只是想要简单开发一个Mod,不需要做什么复杂的处理,如只是扩展一下原作剧情,或是给玩家讲述另一个故事,那么您大概率可以跳过本章关于函数、类的学习。但如果您需要开发更复杂的Mod,比如与DDLC有关的AVG游戏,或是像Monika After Story一样的Mod,那么函数与类无疑会方便您后期的开发。

    但是在学习前,您需要注意一些问题。学习Python的难度也比Ren'Py要大许多。而且在Ren'Py中使用Python代码还要格外小心。如在Python中,有几种数据类型需要格外注意。这些数据类型轻则导致代码出现意料之外的运行结果,重则使Ren'Py不稳定崩溃。

    最后,虽然很多Ren'Py语句都有等效的Python代码,但对于Mod来说,不应该使用Python代替Ren'Py代码。Python的作用是方便开发者进行更复杂的操作而不用修改Ren'Py的底层逻辑。但对于大部分的Mod来说,完全可以使用纯Ren'Py代码。更不用说Python代码在一定程度上会增加脚本的复杂度。Python学好了,用好了,就是锦上添花;如果用不好,就会造成开发难度直线上升,Debug及其困难,还会使游戏崩溃给玩家带来负面体验。
\end{Warning}
\section{Python中的几种数据类型}
\subsection{数字}
在大多数编程语言中,数字可以分为两类——整型(int)与浮点类(float)型。整型顾名思义,就是指3、2、6、-2等不包含小数的数字,浮点型则与之相反,即包含小数的数字,如-2.9、7.1、3.213等。

数字可以进行运算,如加减乘除。在Python中,加法运算使用+号,减法运算使用--号,乘法运算使用*号,除法运算使用/号。对于其他的运算符,详细请见表\ref{table:3.1}。如下例:
\begin{lstlisting}
>>> 1 + 2
%\Output{3}%

>>> 2 / 2
%\Output{1.0}%

>>> 10 - 8
%\Output{2}%

>>> 2 * 7
%\Output{14}%
\end{lstlisting}

\begin{ExtraKnowledge}
    我们可以使用变量将想要将数据保存。使用=操作符对变量进行赋值,如:
    \begin{lstlisting}[numbers=none]
>>> a = 1
>>> a
%\Output{1}%

>>> a = a * 2
>>> a
%\Output{2}%
    \end{lstlisting}
\end{ExtraKnowledge}

对于乘方、整除、取模(余数)计算,则分别使用**运算符、//运算符与\%运算符。如下例:
\begin{lstlisting}
>>> 1 ** 2
%\Output{2}%

>>> 2 ** 3
%\Output{8}%

>>> 10 // 8
%\Output{1}%

>>> 14 // 7
%\Output{2}%

>>> 14 %\%% 7
%\Output{0}%

>>> 25 %\%% 4
%\Output{1}%
\end{lstlisting}

\subsection{字符串}
除了数字,Python还可以处理字符串(str)。在Python中可以使用双引号会单引号括起来表示字符串,也可以使用反斜线操作符对特殊字符转义。

\begin{lstlisting}
>>> 'Hello'
%\Output{'Hello'}%

>>> "Hi!"
%\Output{"Hi!"}%

>>> 'I\'m fine.'
%\Output{"I'm fine."}%

>>> "I'm fine."
%\Output{"I'm fine."}%

>>> print("This is a backslash: \\")
%\Output{This is a backslash: \textbackslash}%

\end{lstlisting}

在Python中,字符串同样支持一些运算功能。+号能将两个字符串连接起来。*则用来重复字符串。如:

\begin{lstlisting}
>>> 'Hello! ' * 3
%\Output{'Hello! Hello! Hello!'}%

>>> "Hi! " + "How are you?"
%\Output{"Hi! How are you?"}%

\end{lstlisting}

\begin{center}
    \tablehead{
    \hline
    操作符 & 描述 & 示例 \\
    \hline
    }
    \tabletail{\hline}
    \tablelasttail{\hline}
    \begin{supertabular}{|c|c|c|}
        \hline
        + & 加法运算,将运算符两侧值相加 & 1 + 2; "str" + "string" \\
        \hline
        - & 减法运算,将运算符两侧值相减 & 1 - 2\\
        \hline
        * & 乘法运算,将操作符两侧值相称 & 4 * 2;"str" * 3\\
        \hline
        / & 除法运算,左操作数除以右操作数 & 3 / 1\\
        \hline
        \% & 取模运算,左操作数除以右操作数的余数部分 & 7 \% 2\\
        \hline
        ** & 幂运算,左操作数的右操作数次幂 & 2 ** 3\\
        \hline
        // & 整除运算,左操作数整除以右操作数& 3 // 2\\
        \hline
        += & 将左侧值加上右侧值 & a += 100\\
        \hline
        -= & 将左侧值减去右侧值 & a -= 100\\
        \hline
        *= & 将左侧值乘上右侧值 & a *= 100\\
        \hline
        /= & 将左侧值除以右侧值 & a /= 100\\
        \hline
    \end{supertabular}
    \captionof{table}{常见算数与赋值运算符与示意}
    \label{table:3.1}
\end{center}


\subsection{列表与元组}

\subsubsection{列表}

在Python中,列表(list)是一种常见的数据类型。在列表可以将多种数据组合在一起。如:
\begin{lstlisting}
>>> ['1', '2', '3']
%\Output{['1', '2', '3']}%

>>> ['This', 'is', 'a', 'list']
%\Output{['This', 'is', 'a', 'list']}%

\end{lstlisting}


在列表中,我们可以使用索引(index)来访问指定数据。如:
\begin{lstlisting}
>>> name_list = ['Sayori', '莫妮卡', 'Yuri', 'Natsuki']
>>> name_list[0]
%\Output{'Sayori'}%

>>> name_list[3]
%\Output{'Natsuki'}%
\end{lstlisting}

\begin{Warning}
    从上例中,我们可以知道索引从0开始。但请注意:索引最大值不可以超出列表的长度。即在name\_list这个列表中,索引最大只能为3,因为此时从0往后数4个数为3。如果索引最大值超出了列表长度,Python就会抛出IndexError错误。如:
    \begin{lstlisting}[numbers=none]
>>> name_list = ['Sayori', 'Monika']
>>> name_list[2]

Traceback (most recent call last):
  File "<stdin>", line 1, in <module>
IndexError: list index out of range
    \end{lstlisting}
\end{Warning}

\begin{ExtraKnowledge}
    索引的值也可以为负数,此时Python会从列表尾部向前寻找索引。但请注意,若要使Python从尾部开始寻找,则索引从-1开始,且同样不可超出列表最大长度。如下例:
    \begin{lstlisting}[numbers=none]
>>> name_list = ['Sayori', 'Monika', 'Natsuki', 'Yuri']
>>> name_list[-1]
%\Output{'Yuri'}%

>>> name_list[-4]
%\Output{'Sayori'}%

>>> name_list[-10]
Traceback (most recent call last):
  File "<stdin>", line 1, in <module>
IndexError: list index out of range
    \end{lstlisting}
\end{ExtraKnowledge}

\begin{ExtraKnowledge}
    索引也可用于字符串,但无法修改字符串的数据。字符串中的索引用法与列表中的索引用法相同。
\end{ExtraKnowledge}

同时,我们可对列表中的数据进行修改。要修改数据,只需要使用索引指定修改的数据,然后使用=重新赋值。如:
\begin{lstlisting}
>>> name_list = ['Sayori', '莫妮卡', 'Natsuki', 'Yuri']
>>> name_list[1] = 'Monika'
>>> name_list

%\Output{['Sayori', 'Monika', 'Natsuki', 'Yuri']}%
\end{lstlisting}

若要添加数据,则可以使用insert函数与append函数。如
\begin{lstlisting}
>>> name_list = ['Monika', 'Sayori']
>>> name_list.insert(1, 'Natsuki')
>>> name_list.append('Yuri')
>>> name_list

%\Output{['Monika', 'Natsuki', 'Sayori', 'Yuri']}%
\end{lstlisting}

对于insert函数,它接受两个参数。第一个参数为一个整数,代表在列表的指定索引出添加一个数据。第二个参数则是添加的数据。

对于append函数,它接受一个参数,即要添加的内容。append函数会在列表末尾添加数据。

若要删除数据,则可以使用pop函数与remove函数。pop函数与remove函数都只接受一个参数。pop函数接受一个整数参数,可以删除列表中指定索引处的数据。remove函数接受一个任意类型的数据,它会先检查列表中是否存在一个数据与参数相同,如果存在则移除,如果不存在则抛出ValueError错误。如:
\begin{lstlisting}
>>> name_list = ['Monika', 'Sayori', 'Yuri', 'Natsuki']
>>> name_list.pop(0)
>>> name_list
%\Output{['Sayori', 'Yuri', 'Natsuki']}%

>>> name_list.remove("Sayori")
%\Output{['Yuri', 'Natsuki']}%

>>> name_list.remove("Monika")
Traceback (most recent call last):
  File "<stdin>", line 1, in <module>
ValueError: list.remove(x): x not in list
\end{lstlisting}

\subsubsection{元组}
元组(tuple)是一种不可变的数据类型。元组支持列表除修改、添加、删除外的所有功能。如:
\begin{lstlisting}
>>> name_list = ('Sayori', 'Monika', 'Natsuki', 'Yuri')
>>> name_list[3]
%\Output{'Yuri'}%

>>> name_list.append("Main Character")
Traceback (most recent call last):
  File "<stdin>", line 1, in <module>
AttributeError: 'tuple' object has no attribute 'append'
\end{lstlisting}

\subsection{字典}
字典(dict)就像它的名字一样,可以像查字典一样取查找。如:
\begin{lstlisting}
>>> point = {'Sayori': 0}
>>> point
%\Output{\{'Sayori': 0\}}%

>>> point['Sayori'] = 2
>>> point['Monika'] = 1
>>> point['Natsuki'] = 3
>>> point['Yuri'] = 1

>>> point
%\Output{\{'Sayori': 2, 'Monika': 1, 'Natsuki': 3, 'Yuri': 1\}}%

>>> point['Sayori']
%\Output{2}%
\end{lstlisting}

\subsection{布尔类型}
布尔类型是最简单的一种类型。布尔类型只包括两个值真(True)、假(False)。如:
\begin{lstlisting}
>>> is_act_two = False
>>> is_act_two
%\Output{False}%

>>> is_act_two = True
>>> is_act_two
%\Output{True}%
\end{lstlisting}

Python中的内置数据类型均可进行逻辑运算和比较。如:
\begin{lstlisting}
>>> a = False
>>> a is True
%\Output{False}%

>>> not a
%\Output{True}%

>>> b = 3
>>> b != 2
%\Output{True}%

>>> c = 2
>>> c <= b
# 即 2 <= 3

%\Output{True}%
\end{lstlisting}
详细逻辑操作符请见表\ref{table:3.1.2}。

\begin{center}
    % \tablefirsthead{
    %     \hline
    %     \multicolumn{1}{|c}{操作符}
    %     \multicolumn{1}{|c|}{描述}
    %     \multicolumn{1}{c|}{示例}
    %     \hline
    % }
    \tablehead{
    \hline
    操作符 & 描述 & 示例\\
    \hline
    }
    \tabletail{\hline}
    \tablelasttail{\hline}
    \captionof{table}{常见逻辑、比较运算符与示意}
    \label{table:3.1.2}
    \begin{supertabular}{|c|p{8cm}|c|}
        \hline
        == & 比较两个对象是否相等 & 2 == 2; a == b\\
        \hline
        != & 比较两个对象是否不等 & 1 != 2; a != b\\
        \hline
        > & 比较左对象是否大于右对象 & 3 > 2\\
        \hline
        < & 比较左对象是否小于右对象 & 2 < 3\\
        \hline
        >= & 比较左对象是否大于等于右对象 & 3 >= 2; 3 >= 3\\
        \hline
        <= & 比较左对象是否小于等于右对象 & 2 <= 3; 3 <= 3\\
        \hline
        is & 比较两个对象内存是否相等(更加严格的==)& a is True\\
        \hline
        not & 逻辑非,用于反转操作数的逻辑状态。即True则为False,False则为True & not True\\
        \hline
        and & 逻辑与,当只有左操作数与右操作数皆为真时,条件为真 & 1 == 2 and 3 < 4\\
        \hline
        or & 逻辑或,当左操作数和右操作数中有一个为真时,条件为真 & 1 == 2 or 10 > 6\\
        \hline
    \end{supertabular}
\end{center}

\begin{ExtraKnowledge}
    此处只对Python的一些常见类型做了简单的介绍。Python中还有其他类型,如可调用类(callable)等。本书目的以教学Ren'Py为主,故不会涉及Python过多。有兴趣者可以前往\url{https://www.runoob.com/python3/python3-tutorial.html}进行更深入的学习。
\end{ExtraKnowledge}


\section{函数与类\PyOnly }

\subsection{函数}
在编程中,我们往往会重复执行一段代码或进行类似的操作。为了减少代码的重复,我们可以使用函数。函数的作用就是把相对独立的某个功能抽象出来,成为一个独立的个体。

\subsubsection{函数的定义}
定义一个函数,只需要开头为def即可。如下例:
\begin{lstlisting}
def test(arg1, arg2):
    print("Arg1 is: " + arg1)
    print("Arg2 is: " + arg2)
    return
\end{lstlisting}

其中,test为这个函数的名字,arg1、arg2则为这个函数接受的参数。若留空,在代表该函数不接受参数。引号后的部分被称为函数主体,是调用该函数后具体的一些代码。return语句则是函数运行成功后返回的值,可以留空。

调用函数也非常简单,如下例:
\begin{lstlisting}
test(1, 2)
%\Output{Arg1 is: 1}%
%\Output{Arg2 is: 2}%

test(5, arg2=1)
%\Output{Arg1 is: 5}%
%\Output{Arg2 is: 1}%

test(arg2=4, arg1=2)
%\Output{Arg1 is: 2}%
%\Output{Arg2 is: 4}%

test(arg1=7, arg2=-7)
%\Output{Arg1 is: 7}%
%\Output{Arg2 is: -7}%
\end{lstlisting}

由此可见,在给函数传递参数时,可以直接传递,也可以使用“参数名=参数”的方式传递。

\subsubsection{函数命名空间}
所有在函数中的变量,都位于一个独立的命名空间内。该命名空间只能在该函数内使用,函数外的代码都无法读取或修改函数命名空间内的变量。同时,函数内也无法直接修改全局命名空间的变量。

如下例:
\begin{lstlisting}
>>> a = 1
>>> def test():
...     a = 2
...     print(a)
>>> print(a)
%\Output{1}%
>>> test()
%\Output{2}%
>>> print(a)
%\Output{1}%
\end{lstlisting}

如果要在函数内修改全局变量,则应使用global语句声明要使用的变量,如下例:
\begin{lstlisting}
>>> a = 1
>>> def test():
...     global a
...     a = 2
...     print(a)


>>> print(a)
%\Output{1}%
>>> test()
%\Output{2}%
>>> print(a)
%\Output{2}%
\end{lstlisting}

\subsection{类}
Python是一门面向对象的语言,而类是一种用来描述具有相同属性和方法(函数)的对象的集合。它定义了该集合中每个对象共同具有的方法。对象是类的实例化。

\subsubsection{类的定义}
定义一个类,只需要开头为class即可。如下例:

\begin{lstlisting}
class Test:
    def __init__(self):
        self.a = 1

    def counter(self):
        self.a += 1
        print(self.a)
\end{lstlisting}

上述例子中,定义了一个名为Test的类,这个类中有一个变量为a,且有一个counter方法用于打印a的值。
\begin{ExtraKnowledge}
    在类中,以双下划线开头的、具有特殊的方法名的方法叫魔法方法(Magic Methods)。上述例子中的\_\_init\_\_特殊方法用于在实例话一个类时会运行的初始化函数。类似的魔法方法还有\_\_eq\_\_,\_\_ne\_\_等。

    self是一个特殊的参数。当类被实例化后,无论调用其中的哪一个方法,Python都会给第一个参数传递这个对象自己,且第一个参数一定指向这个对象。
\end{ExtraKnowledge}

\subsubsection{类的使用}
在使用类前,需要对类进行实例化。实例化后,类就会变成对象。创造对象和创造变量类似。如下例:

\begin{lstlisting}
>>> test = Test()
>>> test.counter()
%\Output{2}%
>>> test.counter()
%\Output{3}%
\end{lstlisting}

第一行的“test = Test()”创造了一个Test对象。剩下两行代码则是在调用这个对象的counter方法。

如果要在类中如果要定义或使用一个属性,必须使用“self.”的方式进行赋值,否则这个属性就只会存在于方法的命名空间而不是对象的命名空间。

\begin{ExtraKnowledge}
    Python中类与函数的使用远远不止这些,您可以前往\url{https://www.runoob.com/python3/python3-function.html}与\url{https://www.runoob.com/python3/python3-class.html}了解更多。
\end{ExtraKnowledge}

\section{在Ren'Py中使用Python语句}
\label{sec:4.3}
在Ren'Py中有多种方式可以使用Python语句:Python语句块(block)、单行Python语句、init python语句。

\subsection{Python语句块}
Python语句块是使用Python的最方便的一种形式。一个Python语句块包含两个部分:
\begin{itemize}
    \item 开头的声明
    \item Python代码
\end{itemize}

如下例:

\begin{lstlisting}
# Chapter 0
label ch0_start:
    python:
        # affection: 好感度
        n_aff = 0
        m_aff = 0
    scene bg club_day
    "{cps=20}快到{b}学园祭{/b}了。{/cps}{w=.5}{nw}"
    show monika 1a at l41 zorder 1
    m "各位!我们得开始准备了!"
    show sayori 1a at h42 zorder 1
    s "好耶!!!!"
    show natsuki 1a at t43 zorder 1
    n 2d "啊,我都等不及学园祭了。"
    n "肯定会很棒的!"
    show yuri 1a at s44 zorder 1
    y "..."
    scene bg club_day
    show monika 2a at t21 zorder 1
    show sayori 2a at t22 zorder 1
    m "那么,是时候来进行分工了。"
    m 4k "[player],你想要做什么? "
    menu:
        "做小蛋糕":
            s "夏树的小蛋糕最好吃了!"
            python:
                $ n_aff += 1
                $ m_aff -= 1
        "布置教室":
            m "那我们可得抓紧时间了!"
            python:
                $ m_aff += 1
                $ n_aff -= 1
    return

\end{lstlisting}

当出现多个Python语句块时,Ren'Py会根据先后顺序依次执行。同时,我们可以使用hide与in关键词来改变Python语句块的行为。

hide关键词会使Python语句块在一个独立的环境(类似于函数命名空间)下运行,即不与其他Python语句块共用一个环境。在具有hide关键词的Python语句块中的所有变量将不会被保存。

in关键词则可以让Python语句块在一个独立的环境下运行。其他环境可以通过“储存区.变量”来使用其他环境的内容。

如下例:
\begin{lstlisting}
python:
    a = 2

python hide:
    a = 1

python in c:
    a = 4

python:
    print(c.a)
    print(a)
\end{lstlisting}

最后的运行结果为:4、2。

\subsection{单行Python语句}
大多数情况下,我们只有一行Python语句需要执行。此时使用Python语句块无疑会对开发者造成多余的输入。为了让编写只有一行的Python更方便快捷,Ren'Py提供了单行Python语句。

单行Python语句以美元符号(\$)开头。如:
\begin{lstlisting}
$ s_aff = 0

$ winner = 'monika'
\end{lstlisting}

\subsection{init python语句}
init python语句在Ren'Py初始化阶段运行,会早于其他代码。这种功能可以用于定义类和函数或者配置变量。
\begin{ExtraKnowledge}
    在init与python之间还可以放一个运行优先级,默认为0。init语句将会按照从低到高的顺序执行。即优先级为-1的代码会优先于优先级为0的代码执行。在优先级相同的情况下,Ren'Py会根据本代码所在文件的Unicode码顺序执行。即在代码优先级都在0的情况下,a.rpy内的代码总会优先于b.rpy内的代码执行。
\end{ExtraKnowledge}

\begin{Attention}
    为了避免与Ren'Py自身代码冲突,您只应使用-999到999范围内作为优先级。同时,原则上除特殊代码外,init优先级应全部大于等于0。
\end{Attention}

init python语句也可以使用hide或in分句,与普通python语句用法相同。

在init python语句中的变量不会被存档。因此,在init语句中定义的应为常量。

\begin{lstlisting}
# Chapter 0
init -1 python:
    demo = True

label ch0_start:
    python:
        # affection: 好感度
        n_aff = 0
        m_aff = 0
    scene bg club_day
    "{cps=20}快到{b}学园祭{/b}了。{/cps}{w=.5}{nw}"
    show monika 1a at l41 zorder 1
    m "各位!我们得开始准备了!"
    show sayori 1a at h42 zorder 1
    s "好耶!!!!"
    show natsuki 1a at t43 zorder 1
    n 2d "啊,我都等不及学园祭了。"
    n "肯定会很棒的!"
    show yuri 1a at s44 zorder 1
    y "..."
    scene bg club_day
    show monika 2a at t21 zorder 1
    show sayori 2a at t22 zorder 1
    m "那么,是时候来进行分工了。"
    m 4k "[player],你想要做什么? "

    if demo:
        "Demo 版剧情到此结束。"
        return

    menu:
        "做小蛋糕":
            s "夏树的小蛋糕最好吃了!"
            python:
                $ n_aff += 1
                $ m_aff -= 1
        "布置教室":
            m "那我们可得抓紧时间了!"
            python:
                $ m_aff += 1
                $ n_aff -= 1
    return

\end{lstlisting}

\begin{ExtraKnowledge}
    Python语句块还可以使用early分句。在Ren'Py的运行生命周期中,python early是被最早运行的。python early中的代码将会早于所有代码运行,因此,python early语句适合用来修改Ren'Py底层处理机制。但由于修改底层处理代码可能会导致一系列问题,我们不会在这里展开讲解。有兴趣者可以前往 \url{https://doc.renpy.cn/zh-CN/lifecycle.html} 了解python early语句的使用。
\end{ExtraKnowledge}


\section{使用变量}
\subsection{python语句块}
关于如何在Python语句块中定义变量,请阅读第\ref{sec:4.3}章
\subsection{define语句}
define语句在初始化时将一个变量赋值。此变量视为一个常量,初始化之后不应再改变。例如:

\begin{lstlisting}
define demo = True
define isActTwo = False
\end{lstlisting}

这段代码的运行效果等于:
\begin{lstlisting}
init python:
    demo = True
    isActTwo = False
\end{lstlisting}

\begin{ExtraKnowledge}
    在define语句中同样可以指定优先级,只需要在define关键词后添加优先级即可,如:define 3 demo = True。
\end{ExtraKnowledge}

define还可以为我们创建一个储存区,只需要将define关键词后的变量改为“储存区.变量”的形式即可。

\subsection{default语句}
default语句会给一个未被定义的变量初始赋值。default语句适合用来定义在游戏过程中会变化的变量。如下例:
\begin{lstlisting}[numbers=none]
default demo = False
\end{lstlisting}

当demo这个变量在游戏开始后没有被定义,则将等价于在start脚本标签中定义demo,且值为False。若在存档加载后没有被定义,则等价于在after\_load魔法标签中定义demo,且值为False。总而言之,若demo这个变量在游戏开始时没有被定义,那么它的值就是False,除非在后来的代码中它的值被改变了。

\subsection{持久化数据}

持久化数据是Ren'Py中的一个储存区,无论用户怎样存档、读档,持久化数据区的数据总是独立于Ren'Py的存档数据的。如DDLC中,对于用户是否在一周目走过了每一条支线、解锁了每一个CG的字典(dict),就存储在持久化数据中,避免因读档存档导致数据消失。简单来说,持久化数据就是不会随着用户存档、读档而改变的数据。

一般来说,在使用DDLC中文Mod模板制作且没有对模版进行设置的游戏,在Ren'Py标准位置里会有一个名为DDLCModTemplateZh的文件夹,里面通常有一个名为persistent的文件,那就是存储持久化数据的存档文件。

持久化数据的用法就和普通变量一样,只不过在前面需要加上“persistent.”前缀。简单来说,持久化数据的语法如下:

\begin{lstlisting}
persistent.<变量名> = <Python 数据类型>
\end{lstlisting}

例如:
\begin{lstlisting}
default persistent.monika_deleted = False
\end{lstlisting}

无论用户如何重启游戏、读档、存档,除非在代码中对持久化数据进行修改或删除persistent文件,persistent.monika\_deleted的值永远都只会是False。
\section{流程控制}
流程控制控制了程序运行的步骤。流程控制包括顺序控制、条件控制和循环控制。顺序控制,顾名思义,就是按照代码的先后顺序,从上到下依次执行代码。

\subsection{脚本标签}
在Ren'Py中,我们可以使用label语句,用自定义的标签名声明一个程序点位。这些标签用于调用或者跳转,可以使用在Ren'Py脚本、python函数及各类界面中。

\subsubsection{label 语句}
在游戏中,故事常常会有多个走向,这是我们就需要编写多个分支。如果剧情只围绕一个分支来讲述故事,那么一定是很枯燥的。同时,如果我们把所有的代码都写在一个label里,无疑会对编写与后期的维护造成不必要的麻烦。这时候,就需要定义多个 label。

label语句的基本语法为:
\begin{lstlisting}
label <标签名>(参数 1, 参数 2):
    <语句 1>
    <语句 2>
    <语句 3>
    ...
\end{lstlisting}

如下例子:
\begin{lstlisting}
label ch0_end:
    scene bg club_day
    "多么美好的一天啊!"
    return
\end{lstlisting}

\begin{Warning}
    通常在label语句末尾,我们都会使用return来返回到上一个调用栈(stack)。在这里您可以理解为回到之前执行的函数、label继续运行游戏。

    如果没有return语句,在执行完本label,Ren'Py会继续调用在本label定义之后的label。
\end{Warning}

label可以在不同的文件内定义。例如我们现在在game目录下创造一个名为script-ch0\_tasks.rpy的文件,并向这个文件中写入以下内容:
\begin{lstlisting}[caption=script-ch0\_tasks.rpy]
label ch0_monika:
    scene bg club_day
    show monika 1a at t11
    m "想好要做什么了吗?"
    return

label ch0_natsuki:
    scene bg club_day
    show natsuki 1a at t11
    n "不过,你知道怎么做小蛋糕吗?"
    return
\end{lstlisting}

此时,我们在ch0\_start标签中可以调用ch0\_monika与ch0\_natsuki标签。

\subsubsection{call语句与jump语句}

现在,让我们修改一下script-ch0.rpy中的内容:
\begin{lstlisting}[caption=script-ch0.rpy]
default n_aff = 0
default s_aff = 0
default demo = False

label ch0_start:
    scene bg club_day
    "{cps=20}快到{b}学园祭{/b}了。{/cps}{w=.5}{nw}"
    show monika 1a at l41 zorder 1
    m "各位!我们得开始准备了!"
    show sayori 1a at h42 zorder 1
    s "好耶!!!!"
    show natsuki 1a at t43 zorder 1
    n 2d "啊,我都等不及学园祭了。"
    n "肯定会很棒的!"
    show yuri 1a at s44 zorder 1
    y "..."
    scene bg club_day
    show monika 2a at t21 zorder 1
    show sayori 2a at t22 zorder 1
    m "那么,是时候来进行分工了。"
    m 4k "[player],你想要做什么? "

    if demo:
        "Demo 版剧情到此结束。"
        return

    menu:
        "做小蛋糕":
            s "夏树的小蛋糕最好吃了!"
            python:
                $ n_aff += 1
                $ m_aff -= 1
            call ch0_natsuki
        "布置教室":
            m "那我们可得抓紧时间了!"
            python:
                $ m_aff += 1
                $ n_aff -= 1
            call ch0_monika
    return
\end{lstlisting}

运行上述代码,我们会发现在玩家做出选择后,执行了在script-ch1\_tasks.rpy中的label中的内容。这就依赖于call语句和jump语句为我们提供的跳转功能了。

call语句和jump语句可以将程序跳转到一个指定的脚本标签处,并且当指定的脚本标签执行完毕后,会自动返回到主控标签继续运行下面的代码。

call语句和jump语句的语法如下:
\begin{lstlisting}
call/jump <标签名>
\end{lstlisting}

或者

\begin{lstlisting}
call/jump expression <label expressions>
\end{lstlisting}

如下例:
\begin{lstlisting}
label main:
    scene bg club_day
    m "你现在正在主标签内。"
    $ today_winner = "sayori"
    call test_natsuki
    m "哦,你回来了?(跳转到 test_natsuki 标签后返回主标签。)"
    call expression "test" + today_winner
    m "你刚刚又去哪里了?(跳转到 test_sayori 标签后再次返回主标签。)"
    jump no_way
    return

label test_natsuki:
    n "你跳转到了 test_natsuki 标签内。"
    return

label test_sayori:
    s "你跳转到了 sub2 标签内。"
    return

label no_way:
    y "好吧,看起来你回不去了。"
    return
\end{lstlisting}

运行上述代码,我们会发现我们一开始会运行main标签中的内容。接着,我们会跳转到test\_natsuki标签内并运行代码,运行完成后我们会返回到main标签中,随后再次跳转到test\_sayori标签内,然后我们又回到了main标签中,最后,我们跳转到了no\_way标签中,并且不再返回main标签,而是回到开始界面。

你或许注意到了,在main中我们并没有直接使用call test\_sayori语句,而是使用了一个简单的表达式,然后把这个运算结果传递给了call。这就是expression选项的作用。使用expression选项,我们不用把标签名写死在程序里,可以立即运算表达式的结果并传递给call。这样做的好处是可以方便地在同一日的多个分支中跳转。

\subsection{if 判断}
大多数游戏中都具有多条剧情线。但面对一些只希望玩家触发多条剧情中的一条时,我们就可以利用if语句判断玩家的剧情线路。在Python和Ren'Py中,if语句的基本语法为:
\begin{lstlisting}
if <表达式>:
    <语句 1>
    <语句 2>
elif <表达式>:
    <语句 3>
else:
    <语句 4>
\end{lstlisting}

每一个if语句中的表达式都应当返回一个True或False(见表\ref{table:4.1.5})。结果为True时,将会执行if语句块中的代码,如果结果为False,Python就会忽略if语句块内的所有代码。

\begin{ExtraKnowledge}
    表达式也可以是一个数字、一个字符串、或定义了\_\_bool\_\_魔术方法的对象。

    不为0的数字、非空的字符串以及\_\_bool\_\_方法返回True的对象都会被视为True
\end{ExtraKnowledge}

如下例:

\begin{lstlisting}
if 1:
    print('1 is True.')

x = True

if x:
    print('x is True.')

if 1 + 1 == 2:
    print('Math is still correct.')
\end{lstlisting}

上述代码的输出结果为:
\begin{lstlisting}
1 is True.
x is True.
Math is still correct.
\end{lstlisting}

当有多个表达式需要同时进行判断或当表达式为False时需要执行一些代码,我们就可以使用elif和else语句。

当if语句中的表达式为False时,会执行else语句的内容。请注意,if、elif或else都必须跟在一起。如下例:

\begin{lstlisting}
if 1 + 1 != 2:
    print('Math crashes!')
elif 1 + 2 == 2:
    print('Math crashes again!')
else:
    print("It's OK. Nothing crazy happened.")
\end{lstlisting}

\subsection{循环}
循环允许我们重复执行一段代码而不需要编写更多的代码。Python中存在两种循环:while循环与for循环。

\subsubsection{while循环}
while循环是Python和Ren'Py中最简单的循环。它的语法结构如下:
\begin{lstlisting}
while <表达式>:
    <语句 1>
    <语句 2>
    ...
\end{lstlisting}

while循环的表达式与if循环的表达式一样。只有表达式为True时才会执行while内的语句。例如:

\begin{lstlisting}
i = 0
while i < 10:
    print(i)
    i += 1
\end{lstlisting}

输出结果为:
\begin{lstlisting}
0
1
2
3
4
5
6
7
8
9
\end{lstlisting}

\begin{Warning}
    请注意,一般表达式不为True,否则就会出现无限循环或死循环。如将上例中的i+= 1删去,就会导致while的表达式始终为True,程序卡死在while循环。在后文中我们将会介绍break和continue语句来打破死循环。
\end{Warning}

\subsubsection{for循环\PyOnly }
for循环比while循环的使用方法更加丰富。它的语法结构如下:
\begin{lstlisting}
for <变量名> in <序列>:
    <语句 1>
    <语句 2>
    ...
\end{lstlisting}

这里的序列可以是列表、元组等可迭代对象。当序列中不再有变量后,for循环会停止运行。如下例:

\begin{lstlisting}
t1 = ('Sayori', 'Monika', 'Yuri', 'Natsuki')
l1 = [0, 0, 2, 0]

for i in t1:
    print(i)

for i in l1:
    print(i)
\end{lstlisting}

输出结果为:
\begin{lstlisting}
Sayori
Monika
Yuri
Natsuki
0
0
2
0
\end{lstlisting}

\begin{ExtraKnowledge}
    range函数是Python的内部函数之一,它可以为我们生成一个生成器(类似于列表,但比列表的性能更好)。使用range函数,我们可以快速生成一个从0开始到某数结束的一个生成器。当然,我们也可以指定range函数的起始数字与结束数字以及步长。如下例:
    \begin{lstlisting}
    for i in range(10):
        print(i)

    print("================================================")

    for i in range(10, 0, -1):
        print(i)
    \end{lstlisting}
    输出结果为:
    \begin{lstlisting}
    0
    1
    2
    3
    4
    5
    6
    7
    8
    9
    ================================================
    10
    9
    8
    7
    6
    5
    4
    3
    2
    1
    \end{lstlisting}

    上述第二个例子中的10就是起始数字,0则为截止数字,-1就是步长。
\end{ExtraKnowledge}

在部分情况中,我们希望可以跳过循环或退出循环体,这时我们就可以使用break和continue语句了。break可以立即退出循环体,如下例:

\begin{lstlisting}
for i in range(10):
    print(i)
    if i > 5:
        break
\end{lstlisting}

输出结果为:
\begin{lstlisting}
0
1
2
3
4
5
\end{lstlisting}

continue可以跳过当前的循环,如下例:

\begin{lstlisting}
for i in range(10):
    print(i)
    if i == 5:
        continue
\end{lstlisting}


输出结果为:
\begin{lstlisting}
0
1
2
3
4
6
7
8
9
\end{lstlisting}

同时,在Python中,循环也可以使用else语句。在while语句中的else语句会在while表达式为False时被执行。如下例:

\begin{lstlisting}
i = 0
while i <= 9:
    print(i)
    i += 1
else:
    print(i, " is bigger than 9")
\end{lstlisting}

输出结果为:
\begin{lstlisting}[language=C]
0
1
2
3
4
5
6
7
8
9
10 is bigger than 9
\end{lstlisting}

同时,请注意break导致的循环体退出不会执行else语句中的内容。

\subsection{错误和异常\PyOnly }
在Python中,不正常或语法错误的代码将会抛出异常。异常会使程序停止运行、崩溃、闪退等。常见的异常有:NameError(尝试使用一个未定义的变量)、IndexError(尝试访问在列表或元组范围外的索引)、TypeError(试图将两个不支持运算的类型进行运算)等。

为了处理这些异常,我们可以使用try-except语句。它的语法结构如下:
\begin{lstlisting}
try:
    <可能抛出错误的语句>
except <错误类型>:
    <当错误被捕获后的语句>
\end{lstlisting}

如下例子:

\begin{lstlisting}
try:
    x = 1 / 0
except ZeroDivisionError:
    print('Error: Cannot divide by zero.')
\end{lstlisting}

输出结果为:
\begin{lstlisting}
Error: Cannot divide by zero.
\end{lstlisting}

进阶的语法包含finally从句或else从句。请注意,finally从句与else从句不可并存。

在try-except-finally中,无论try代码块中的代码是否抛出异常,finally从句中的代码都一定会被执行。try-except-finally的语法如下:
\begin{lstlisting}
try:
    <可能抛出错误的语句>
except <错误类型>:
    <当错误被捕获后的语句>
finally:
    <无论是否抛出错误都会执行的语句>
\end{lstlisting}

如下例子:

\begin{lstlisting}
div = 0
def func():
    global div
    return 1 / div

try:
    x = func()
except ZeroDivisionError:
    print('Error: Cannot divide by zero.')
finally:
    x = 10

print(x)

div = 10
try:
    x = func()
except ZeroDivisionError:
    print('Error: Cannot divide by zero.')
finally:
    x = 100
print(x)
\end{lstlisting}

输出结果为:
\begin{lstlisting}
Error: Cannot divide by zero.
10
100
\end{lstlisting}

try-except-else中,只有当try语句块中的内容没有抛出异常时,else中的内容才会被执行。try-except-else的语法如下:
\begin{lstlisting}
try:
    <可能抛出错误的语句>
except <错误类型>:
    <当错误被捕获后的语句>
else:
    <当没有错误被捕获时执行的语句>
\end{lstlisting}

如下例:

\begin{lstlisting}
try:
    x = 1 / 0
except ZeroDivisionError:
    print('Error: Cannot divide by zero.')
else:
    print("The value of x is ", x)

try:
    x = 1 / 10
except ZeroDivisionError:
    print('Error: Cannot divide by zero.')
else:
    print("The value of x is ", x)
\end{lstlisting}

输出结果为:

\begin{lstlisting}
Error: Cannot divide by zero.
The value of x is 0.1
\end{lstlisting}
\section{等效语句}
等效语句是在Python中使用Ren'Py语法的一种方式。等效语句只能在Python代码中运行。

\subsection{对话}
Ren'Py的say语句对应两种等效语句。如下例:

\begin{lstlisting}
m "你好!"
\end{lstlisting}

这段代码不仅等效于下列代码:

\begin{lstlisting}
$ m("你好!")
\end{lstlisting}

同时也等效于:

\begin{lstlisting}
$ renpy.say(m, "你好")
\end{lstlisting}

不过,为了保持语义上的通畅,我们常使用后者。

\subsection{图像显示}

\subsubsection{show}

show语句的等效语句的语法如下:

\begin{lstlisting}
renpy.show(<name>, at_list=<position>, zorder=<zorder number>)
\end{lstlisting}

\subsubsection{hide}

hide语句的等效语句的语法如下:

\begin{lstlisting}
renpy.hide(<name>)
\end{lstlisting}

\subsubsection{scene}

scene语句的等效语句的语法如下:

\begin{lstlisting}
renpy.scene()
renpy.show(<name>)
\end{lstlisting}

\subsubsection{with 从句}

with 语句的等效语句的语法如下:

\begin{lstlisting}
renpy.with_statement(<name>)
\end{lstlisting}

\subsection{call 和 jump}

call 和 jump 语句的等效语句的语法如下:

\begin{lstlisting}
renpy.call(script=<script name>)
renpy.jump(screen=<script name>)
\end{lstlisting}

对于更多关于等效语句的介绍,请查阅 \url{https://doc.renpy.cn/zh-CN/statement_equivalents.html}