\chapter{Python与Ren'Py}
\begin{ChapterGoals}
    \begin{itemize}
        \item 了解Python的基本数据类型;
        \item 学会使用函数与类;
        \item 学会在Ren'Py代码中嵌入Python代码;
        \item 学会使用变量;
        \item 学会进行判断控制;
        \item 学会使用等效语句;
    \end{itemize}
\end{ChapterGoals}

Ren'Py与Python是一个不可分割的整体,在Ren'Py中,许多复杂的操作都需要依靠Python来完成。本章我们将会初步学习如何在Ren'Py中使用Python语句。

\begin{Attention}
    本章只会浅显的介绍Python的用法,若要真正的学习Python,您可以前往\url{https://www.runoob.com/python3/python3-tutorial.html}进行更深入的学习。切记,本书不以Python为主。
\end{Attention}

\begin{Warning}
    如果您只是想要简单开发一个Mod,不需要做什么复杂的处理,如只是扩展一下原作剧情,或是开发另一个故事,那么您大概率可以跳过本章的学习。但如果您需要开发更复杂的Mod,比如与DDLC有关的AVG游戏,或是像Monika After Story一样的Mod,那么学习本章无疑会方便您后期的开发。

    但是在学习前,您需要注意一些问题。学习Python的难度也比Ren'Py要大许多。而且在Ren'Py中使用Python代码还要格外小心。如在Python中,有几种数据类型需要格外注意。这些数据类型轻则导致代码出现意料之外的运行结果,重则使Ren'Py崩溃。

    最后,虽然很多Ren'Py语句都有等效的Python代码,但对于Mod来说,不应该使用Python代替Ren'Py代码。Python的作用是方便开发者进行更复杂的操作,如设置对话回调函数。但对于大部分的Mod来说,完全可以使用纯Ren'Py代码。且Python代码在一定程度上会增加脚本的复杂度。Python学好了,用好了,就是锦上添花;如果用不好,就会造成开发难度直线上升,并且可能会使Ren'Py崩溃。

    请记住,不到万不可以的时候,尽量不要使用Python代码。
\end{Warning}
\section{增添资源}
一般来说,在Mod工程里的game文件夹下会有一个名叫mod\_assets的文件夹。所有在mod\_assets文件夹里的内容都会在生成发行版时被打包成一个rpa文件。(在后续的章节里,我们将会详细学习如何控制哪些文件打包、哪些不打包,以及如何生成发行版)

\begin{ExtraKnowledge}
    目前,常见的资源获取方法有:从DDLC Community Assets获取(Google Drive)、从一些开发QQ群内获取打包好的DDLC Communi Assets、Reddit等。
\end{ExtraKnowledge}

\begin{Attention}
    在使用资源前,请务必注意版权问题。一般来说,在DDLC Community Assets中的文件只需要在感谢名单中添加作者的名字,但也有部分资源会有更多的要求。请记住:在一个资源必须得到作者授权的情况下,没有得到授权就使用该资源的行为是侵权行为。
\end{Attention}


\subsection{定义角色}
在一些模组中,我们常常需要添加一些角色来丰富故事线,或是推动故事情节的发展。要增加角色,我们需要先了解如何定义一个角色。

定义一个角色有两种方式:Character与DynamicCharacter。两种定义方式几乎没有区别,唯一的区别在于Character的名字是固定的,无法更改的;而DynamicCharacter则使用一个变量作为角色名,是动态的,可以更改。由于DDLC中角色都使用DynamicCharacter来定义角色,故本书不会介绍如何使用Character定义,感兴趣者可以前往Ren'Py中文文档了解。

DynamicCharacter的语法如下:

\begin{lstlisting}
define <变量名> = DynamicCharacter('<存储角色名的变量名>', image='<say语句的对话属性图像名>', what_prefix='"', what_suffix='"', ctc="ctc", ctc_position="fixed")
\end{lstlisting}

\begin{Warning}
    对于say语句的对话属性来说,如果image后的图像名错误,那么将会导致其无法使用。所以请务必确保image后的图像名与您定义的角色立绘名一致。对于角色立绘名,请参考\ref{par:3.2.1}
\end{Warning}

\begin{ExtraKnowledge}
    在某些情况下,我们可能改变角色说话内容的双引号为其他符号,我们可以通过修改what\_prefix和what\_suffix实现效果。

    正常情况下,除旁白外所有的角色说的话都会被双引号包住。如:
    \begin{lstlisting}
m "你好"
    \end{lstlisting}

    那么在游戏中的效果为:"你好"。如果我们想要修改为:【你好】,那么只需要找到莫妮卡角色的定义,并将what\_prefix改为'【', what\_suffix改为'】'即可实现。实际代码如下:
    \begin{lstlisting}
define m = DynamicCharacter('m_name', image='monika', what_prefix='【', what_suffix='】', ctc='ctc', ctc_position='fixed')
    \end{lstlisting}
\end{ExtraKnowledge}

如我现在需要定义一个名为Charlie的角色,其使用的图像名为charlie,存储角色名的变量名叫c\_name:
\begin{lstlisting}
define c_name = "Charlie"
define c = DynamicCharacter('c_name', image='charlie', what_prefix='"', what_suffix='"', ctc="ctc", ctc_position="fixed")
\end{lstlisting}

现在,我们就成功定义了Charlie这个角色。那么这段代码应该放在哪里呢?答案是game目录下的definitions.rpy文件内。这个文件里储存着所有在游戏中需要用到的资源的定义。

打开编辑器的查找功能,搜索:“define s = DynamicCharacter”,随后在这一行上面或下面增添例如上述的代码。

现在,definitions.rpy文件内的代码应该长这样:
\begin{lstlisting}
    # 角色变量

    define narrator = Character(ctc="ctc", ctc_position="fixed")
    define mc = DynamicCharacter('player', what_prefix='"', what_suffix='"', ctc="ctc", ctc_position="fixed")
    define s = DynamicCharacter('s_name', image='sayori', what_prefix='"', what_suffix='"', ctc="ctc", ctc_position="fixed")
    define m = DynamicCharacter('m_name', image='monika', what_prefix='"', what_suffix='"', ctc="ctc", ctc_position="fixed")
    define n = DynamicCharacter('n_name', image='natsuki', what_prefix='"', what_suffix='"', ctc="ctc", ctc_position="fixed")
    define y = DynamicCharacter('y_name', image='yuri', what_prefix='"', what_suffix='"', ctc="ctc", ctc_position="fixed")
    define ny = Character('夏树 & 优里', what_prefix='"', what_suffix='"', ctc="ctc", ctc_position="fixed")
    define c = DynamicCharacter('c_name', image='charlie', what_prefix='"', what_suffix='"', ctc="ctc", ctc_position="fixed")
    
    # ...
    
    # Default Name Variables
    default s_name = "纱世里"
    default m_name = "莫妮卡"
    default n_name = "夏树"
    default y_name = "优里"
    default a_name = "Charlie"
    
    
\end{lstlisting}

\subsection{增加、定义图片}

现在,在mod\_assets文件夹下创造一个名为images的文件夹。在后续的教程中,我们将会把所有的图片资源都存储在这个images文件夹内。

\begin{Warning}
图像资源的格式应该为PNG,且背景应该是透明的。
角色图像资源的分辨率应为 960x960 以保证兼容性。
背景图像的分辨率应该为 1280x720,否则图片在显示时会出现意料之外的状况。如果背景图片的分辨率超过上述分辨率,至少您的图片也应该为 16:9 的尺寸,这种情况下可以使用增添size属性解决问题:

\begin{lstlisting}
size (1280,720) # 添加 size 属性
\end{lstlisting}

为了兼容性,我们建议您新建的所有文件(夹)名称全部为英文小写字母,且不使用中文。
\end{Warning}

\subsubsection{角色图片}

在获取角色资源图片后,

\section{函数与类\PyOnly }

\subsection{函数}
在编程中,我们往往会重复执行一段代码或进行类似的操作。为了减少代码的重复,我们可以使用函数。函数的作用就是把相对独立的某个功能抽象出来,成为一个独立的个体。

\subsubsection{函数的定义}
定义一个函数,只需要开头为def即可。如下例:
\begin{lstlisting}
def test(arg1, arg2):
    print("Arg1 is: " + arg1)
    print("Arg2 is: " + arg2)
    return
\end{lstlisting}

其中,test为这个函数的名字,arg1、arg2则为这个函数接受的参数。若留空,在代表该函数不接受参数。引号后的部分被称为函数主体,是调用该函数后具体的一些代码。return语句则是函数运行成功后返回的值,可以留空。

调用函数也非常简单,如下例:
\begin{lstlisting}
test(1, 2)
%\Output{Arg1 is: 1}%
%\Output{Arg2 is: 2}%

test(5, arg2=1)
%\Output{Arg1 is: 5}%
%\Output{Arg2 is: 1}%

test(arg2=4, arg1=2)
%\Output{Arg1 is: 2}%
%\Output{Arg2 is: 4}%

test(arg1=7, arg2=-7)
%\Output{Arg1 is: 7}%
%\Output{Arg2 is: -7}%
\end{lstlisting}

由此可见,在给函数传递参数时,可以直接传递,也可以使用“参数名=参数”的方式传递。

\subsubsection{函数命名空间}
所有在函数中的变量,都位于一个独立的命名空间内。该命名空间只能在该函数内使用,函数外的代码都无法读取或修改函数命名空间内的变量。同时,函数内也无法直接修改全局命名空间的变量。

如下例:
\begin{lstlisting}
>>> a = 1
>>> def test():
...     a = 2
...     print(a)
>>> print(a)
%\Output{1}%
>>> test()
%\Output{2}%
>>> print(a)
%\Output{1}%
\end{lstlisting}

如果要在函数内修改全局变量,则应使用global语句声明要使用的变量,如下例:
\begin{lstlisting}
>>> a = 1
>>> def test():
...     global a
...     a = 2
...     print(a)


>>> print(a)
%\Output{1}%
>>> test()
%\Output{2}%
>>> print(a)
%\Output{2}%
\end{lstlisting}

\subsection{类}
Python是一门面向对象的语言,而类是一种用来描述具有相同属性和方法(函数)的对象的集合。它定义了该集合中每个对象共同具有的方法。对象是类的实例化。

\subsubsection{类的定义}
定义一个类,只需要开头为class即可。如下例:

\begin{lstlisting}
class Test:
    def __init__(self):
        self.a = 1

    def counter(self):
        self.a += 1
        print(self.a)
\end{lstlisting}

上述例子中,定义了一个名为Test的类,这个类中有一个变量为a,且有一个counter方法用于打印a的值。
\begin{ExtraKnowledge}
    在类中,以双下划线开头的、具有特殊的方法名的方法叫魔法方法(Magic Methods)。上述例子中的\_\_init\_\_特殊方法用于在实例话一个类时会运行的初始化函数。类似的魔法方法还有\_\_eq\_\_,\_\_ne\_\_等。

    self是一个特殊的参数。当类被实例化后,无论调用其中的哪一个方法,Python都会给第一个参数传递这个对象自己,且第一个参数一定指向这个对象。
\end{ExtraKnowledge}

\subsubsection{类的使用}
在使用类前,需要对类进行实例化。实例化后,类就会变成对象。创造对象和创造变量类似。如下例:

\begin{lstlisting}
>>> test = Test()
>>> test.counter()
%\Output{2}%
>>> test.counter()
%\Output{3}%
\end{lstlisting}

第一行的“test = Test()”创造了一个Test对象。剩下两行代码则是在调用这个对象的counter方法。

如果要在类中如果要定义或使用一个属性,必须使用“self.”的方式进行赋值,否则这个属性就只会存在于方法的命名空间而不是对象的命名空间。

\begin{ExtraKnowledge}
    Python中类与函数的使用远远不止这些,您可以前往\url{https://www.runoob.com/python3/python3-function.html}与\url{https://www.runoob.com/python3/python3-class.html}了解更多。
\end{ExtraKnowledge}

\input{Chapters/3/3}
\input{Chapters/3/4}
\section{流程控制}
流程控制控制了程序运行的步骤。流程控制包括顺序控制、条件控制和循环控制。顺序控制,顾名思义,就是按照代码的先后顺序,从上到下依次执行代码。

\subsection{脚本标签}
在Ren'Py中,我们可以使用label语句,用自定义的标签名声明一个程序点位。这些标签用于调用或者跳转,可以使用在Ren'Py脚本、python函数及各类界面中。

\subsubsection{label 语句}
在游戏中,故事常常会有多个走向,这是我们就需要编写多个分支。如果剧情只围绕一个分支来讲述故事,那么一定是很枯燥的。同时,如果我们把所有的代码都写在一个label里,无疑会对编写与后期的维护造成不必要的麻烦。这时候,就需要定义多个 label。

label语句的基本语法为:
\begin{lstlisting}
label <标签名>(参数 1, 参数 2):
    <语句 1>
    <语句 2>
    <语句 3>
    ...
\end{lstlisting}

如下例子:
\begin{lstlisting}
label ch0_end:
    scene bg club_day
    "多么美好的一天啊!"
    return
\end{lstlisting}

\begin{Warning}
    通常在label语句末尾,我们都会使用return来返回到上一个调用栈(stack)。在这里您可以理解为回到之前执行的函数、label继续运行游戏。

    如果没有return语句,在执行完本label,Ren'Py会继续调用在本label定义之后的label。
\end{Warning}

label可以在不同的文件内定义。例如我们现在在game目录下创造一个名为script-ch0\_tasks.rpy的文件,并向这个文件中写入以下内容:
\begin{lstlisting}[caption=script-ch0\_tasks.rpy]
label ch0_monika:
    scene bg club_day
    show monika 1a at t11
    m "想好要做什么了吗?"
    return

label ch0_natsuki:
    scene bg club_day
    show natsuki 1a at t11
    n "不过,你知道怎么做小蛋糕吗?"
    return
\end{lstlisting}

此时,我们在ch0\_start标签中可以调用ch0\_monika与ch0\_natsuki标签。

\subsubsection{call语句与jump语句}

现在,让我们修改一下script-ch0.rpy中的内容:
\begin{lstlisting}[caption=script-ch0.rpy]
default n_aff = 0
default s_aff = 0
default demo = False

label ch0_start:
    scene bg club_day
    "{cps=20}快到{b}学园祭{/b}了。{/cps}{w=.5}{nw}"
    show monika 1a at l41 zorder 1
    m "各位!我们得开始准备了!"
    show sayori 1a at h42 zorder 1
    s "好耶!!!!"
    show natsuki 1a at t43 zorder 1
    n 2d "啊,我都等不及学园祭了。"
    n "肯定会很棒的!"
    show yuri 1a at s44 zorder 1
    y "..."
    scene bg club_day
    show monika 2a at t21 zorder 1
    show sayori 2a at t22 zorder 1
    m "那么,是时候来进行分工了。"
    m 4k "[player],你想要做什么? "

    if demo:
        "Demo 版剧情到此结束。"
        return

    menu:
        "做小蛋糕":
            s "夏树的小蛋糕最好吃了!"
            python:
                $ n_aff += 1
                $ m_aff -= 1
            call ch0_natsuki
        "布置教室":
            m "那我们可得抓紧时间了!"
            python:
                $ m_aff += 1
                $ n_aff -= 1
            call ch0_monika
    return
\end{lstlisting}

运行上述代码,我们会发现在玩家做出选择后,执行了在script-ch1\_tasks.rpy中的label中的内容。这就依赖于call语句和jump语句为我们提供的跳转功能了。

call语句和jump语句可以将程序跳转到一个指定的脚本标签处,并且当指定的脚本标签执行完毕后,会自动返回到主控标签继续运行下面的代码。

call语句和jump语句的语法如下:
\begin{lstlisting}
call/jump <标签名>
\end{lstlisting}

或者

\begin{lstlisting}
call/jump expression <label expression>
\end{lstlisting}

如下例:
\begin{lstlisting}
label main:
    scene bg club_day
    m "你现在正在主标签内。"
    $ today_winner = "sayori"
    call test_natsuki
    m "哦,你回来了?(跳转到 test_natsuki 标签后返回主标签。)"
    call expression "test" + today_winner
    m "你刚刚又去哪里了?(跳转到 test_sayori 标签后再次返回主标签。)"
    jump no_way
    return

label test_natsuki:
    n "你跳转到了 test_natsuki 标签内。"
    return

label test_sayori:
    s "你跳转到了 sub2 标签内。"
    return

label no_way:
    y "好吧,看起来你回不去了。"
    return
\end{lstlisting}

运行上述代码,我们会发现我们一开始会运行main标签中的内容。接着,我们会跳转到test\_natsuki标签内并运行代码,运行完成后我们会返回到main标签中,随后再次跳转到test\_sayori标签内,然后我们又回到了main标签中,最后,我们跳转到了no\_way标签中,并且不再返回main标签,而是回到开始界面。

你或许注意到了,在main中我们并没有直接使用call test\_sayori语句,而是使用了一个简单的表达式,然后把这个运算结果传递给了call。这就是expression选项的作用。使用expression选项,我们不用把标签名写死在程序里,可以立即运算表达式的结果并传递给call。这样做的好处是可以方便地在同一日的多个分支中跳转。

\subsection{if 判断}
大多数游戏中都具有多条剧情线。但面对一些只希望玩家触发多条剧情中的一条时,我们就可以利用if语句判断玩家的剧情线路。在Python和Ren'Py中,if语句的基本语法为:
\begin{lstlisting}
if <表达式>:
    <语句 1>
    <语句 2>
elif <表达式>:
    <语句 3>
else:
    <语句 4>
\end{lstlisting}

每一个if语句中的表达式都应当返回一个True或False(见表\ref{table:3.1.2})。结果为True时,将会执行if语句块中的代码,如果结果为False,Python就会忽略if语句块内的所有代码。

\begin{ExtraKnowledge}
    表达式也可以是一个数字、一个字符串、或定义了\_\_bool\_\_魔术方法的对象。

    不为0的数字、非空的字符串以及\_\_bool\_\_方法返回True的对象都会被视为True
\end{ExtraKnowledge}

如下例:

\begin{lstlisting}
if 1:
    print('1 is True.')

x = True

if x:
    print('x is True.')

if 1 + 1 == 2:
    print('Math is still correct.')
\end{lstlisting}

上述代码的输出结果为:
\begin{lstlisting}
1 is True.
x is True.
Math is still correct.
\end{lstlisting}

当有多个表达式需要同时进行判断或当表达式为False时需要执行一些代码,我们就可以使用elif和else语句。

当if语句中的表达式为False时,会执行else语句的内容。请注意,if、elif或else都必须跟在一起。如下例:

\begin{lstlisting}
if 1 + 1 != 2:
    print('Math crashes!')
elif 1 + 2 == 2:
    print('Math crashes again!')
else:
    print("It's OK. Nothing crazy happened.")
\end{lstlisting}

\subsection{循环}
循环允许我们重复执行一段代码而不需要编写更多的代码。Python中存在两种循环:while循环与for循环。

\subsubsection{while循环}
while循环是Python和Ren'Py中最简单的循环。它的语法结构如下:
\begin{lstlisting}
while <表达式>:
    <语句 1>
    <语句 2>
    ...
\end{lstlisting}

while循环的表达式与if循环的表达式一样。只有表达式为True时才会执行while内的语句。例如:

\begin{lstlisting}
i = 0
while i < 10:
    print(i)
    i += 1
\end{lstlisting}

输出结果为:
\begin{lstlisting}
0
1
2
3
4
5
6
7
8
9
\end{lstlisting}

\begin{Warning}
    请注意,一般表达式不为True,否则就会出现无限循环或死循环。如将上例中的i+= 1删去,就会导致while的表达式始终为True,程序卡死在while循环。在后文中我们将会介绍break和continue语句来打破死循环。
\end{Warning}

\subsubsection{for循环\PyOnly }
for循环比while循环的使用方法更加丰富。它的语法结构如下:
\begin{lstlisting}
for <Variable> in <Sequence>:
    <语句 1>
    <语句 2>
    ...
\end{lstlisting}

这里的序列可以是列表、元组等可迭代对象。当序列中不再有变量后,for循环会停止运行。如下例:

\begin{lstlisting}
t1 = ('Sayori', 'Monika', 'Yuri', 'Natsuki')
l1 = [0, 0, 2, 0]

for i in t1:
    print(i)

for i in l1:
    print(i)
\end{lstlisting}

输出结果为:
\begin{lstlisting}
Sayori
Monika
Yuri
Natsuki
0
0
2
0
\end{lstlisting}

\begin{ExtraKnowledge}
    range函数是Python的内部函数之一,它可以为我们生成一个生成器(类似于列表,但比列表的性能更好)。使用range函数,我们可以快速生成一个从0开始到某数结束的一个生成器。当然,我们也可以指定range函数的起始数字与结束数字以及步长。如下例:
    \begin{lstlisting}
    for i in range(10):
        print(i)

    print("================================================")

    for i in range(10, 0, -1):
        print(i)
    \end{lstlisting}
    输出结果为:
    \begin{lstlisting}
    0
    1
    2
    3
    4
    5
    6
    7
    8
    9
    ================================================
    10
    9
    8
    7
    6
    5
    4
    3
    2
    1
    \end{lstlisting}

    上述第二个例子中的10就是起始数字,0则为截止数字,-1就是步长。
\end{ExtraKnowledge}

在部分情况中,我们希望可以跳过循环或退出循环体,这时我们就可以使用break和continue语句了。break可以立即退出循环体,如下例:

\begin{lstlisting}
for i in range(10):
    print(i)
    if i > 5:
        break
\end{lstlisting}

输出结果为:
\begin{lstlisting}
0
1
2
3
4
5
\end{lstlisting}

continue可以跳过当前的循环,如下例:

\begin{lstlisting}
for i in range(10):
    print(i)
    if i == 5:
        continue
\end{lstlisting}


输出结果为:
\begin{lstlisting}
0
1
2
3
4
6
7
8
9
\end{lstlisting}

同时,在Python中,循环也可以使用else语句。在while语句中的else语句会在while表达式为False时被执行。如下例:

\begin{lstlisting}
i = 0
while i <= 9:
    print(i)
    i += 1
else:
    print(i, " is bigger than 9")
\end{lstlisting}

输出结果为:
\begin{lstlisting}[language=C]
0
1
2
3
4
5
6
7
8
9
10 is bigger than 9
\end{lstlisting}

同时,请注意break导致的循环体退出不会执行else语句中的内容。

\subsection{错误和异常\PyOnly }
在Python中,不正常或语法错误的代码将会抛出异常。异常会使程序停止运行、崩溃、闪退等。常见的异常有:NameError(尝试使用一个未定义的变量)、IndexError(尝试访问在列表或元组范围外的索引)、TypeError(试图将两个不支持运算的类型进行运算)等。

为了处理这些异常,我们可以使用try-except语句。它的语法结构如下:
\begin{lstlisting}
try:
    <可能抛出错误的语句>
except <错误类型>:
    <当错误被捕获后的语句>
\end{lstlisting}

如下例子:

\begin{lstlisting}
try:
    x = 1 / 0
except ZeroDivisionError:
    print('Error: Cannot divide by zero.')
\end{lstlisting}

输出结果为:
\begin{lstlisting}
Error: Cannot divide by zero.
\end{lstlisting}

进阶的语法包含finally从句或else从句。请注意,finally从句与else从句不可并存。

在try-except-finally中,无论try代码块中的代码是否抛出异常,finally从句中的代码都一定会被执行。try-except-finally的语法如下:
\begin{lstlisting}
try:
    <可能抛出错误的语句>
except <错误类型>:
    <当错误被捕获后的语句>
finally:
    <无论是否抛出错误都会执行的语句>
\end{lstlisting}

如下例子:

\begin{lstlisting}
div = 0
def func():
    global div
    return 1 / div

try:
    x = func()
except ZeroDivisionError:
    print('Error: Cannot divide by zero.')
finally:
    x = 10

print(x)

div = 10
try:
    x = func()
except ZeroDivisionError:
    print('Error: Cannot divide by zero.')
finally:
    x = 100
print(x)
\end{lstlisting}

输出结果为:
\begin{lstlisting}
Error: Cannot divide by zero.
10
100
\end{lstlisting}

try-except-else中,只有当try语句块中的内容没有抛出异常时,else中的内容才会被执行。try-except-else的语法如下:
\begin{lstlisting}
try:
    <可能抛出错误的语句>
except <错误类型>:
    <当错误被捕获后的语句>
else:
    <当没有错误被捕获时执行的语句>
\end{lstlisting}

如下例:

\begin{lstlisting}
try:
    x = 1 / 0
except ZeroDivisionError:
    print('Error: Cannot divide by zero.')
else:
    print("The value of x is ", x)

try:
    x = 1 / 10
except ZeroDivisionError:
    print('Error: Cannot divide by zero.')
else:
    print("The value of x is ", x)
\end{lstlisting}

输出结果为:

\begin{lstlisting}
Error: Cannot divide by zero.
The value of x is 0.1
\end{lstlisting}
\input{Chapters/3/6}