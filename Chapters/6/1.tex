\section{修改设置}

在game文件夹下,有一个options.rpy文件,在其中定义了一些与Mod个性化的一些设置。在打包您的模组前,我们需要修改一些设置来保证模组不会与其他模组冲突。

现在,打开options.rpy文件。找到以下代码:
\begin{lstlisting}
    define config.name = "DDLC 中文 Mod 模板"
    define config.version = "2.0.0-dev"
    define gui.about = _("""这里是写简介的地方。在 options.rpy 里写上你的 Mod 简介吧!""")
    define build.name = "DDLCModTempCNNext"
    define config.save_directory = "DDLCModTempCNNext"
\end{lstlisting}

按照以下内容修改上述代码:
\begin{enumerate}
    \item 将config.name的内容修改为您的模组的名字;
    \item 将config.version修改为"1.0.0"(您也可以自定义版本号);
    \item 将gui.about修改为您模组的简介;
    \item 将build.name修改为您模组的名字;
    \begin{Attention}
        请注意只能使用英文字母,中文、下划线等会导致打包失败。
    \end{Attention}
    \item 将config.save\_directory修改为您模组的名字。
    \item \begin{Attention}
        请注意只能使用英文字母,中文、下划线等会导致游戏无法打开、持久化数据无法保存等问题。
    \end{Attention}
\end{enumerate}
如:
\begin{lstlisting}
    define config.name = "Monika 大战 360" #???
    define config.version = "1.0.0"
    define gui.about = _("""Monika想要删除文件,可是万恶的360却拦住了她。接下来会发生什么呢...""")
    define build.name = "MonikaFightWithThreeSixZero"
    define config.save_directory = "MonikaFightWithThreeSixZero"
\end{lstlisting}

然后,打开definitions.rpy文件,找到以下代码:
\begin{lstlisting}
    define persistent.demo = False
    define config.developer = False
\end{lstlisting}

确保上述两个常量为False。

完成后,您现在可以打包您的模组了。